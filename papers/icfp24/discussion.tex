\section{Discussion}
\label{sec:discussion}

To conclude, our framework of universal algebra can be generalised from sets to groupoids, using a system of
coherences on top of the system of equations.

Hinze's work: Sorting and Searching by Distribution: From Generic Discrimination to Generic Trie, see 4.6 on page 324


\vc{Can you sort anything that doesn't have a total order?}
\vc{Can you sort binary trees?}

In a sense, this problem has already been solved, first by Hinze et al. \cite{hinzeSortingBialgebrasDistributive2012}
and later extended by Alexandru in their thesis \cite{alexandruIntrinsicallyCorrectSorting2023}.
Their formalization is defined in terms of bialgebras, which not
only captures the correctness of sorting algorithms purely in a categorical settings, but
also isolate the computational essence of sorting algorithms in terms of distributive laws,
allowing us to construct more sorting algorithms "for free". Their work are thus necessarily
below the level of extensional equality, i.e. input-output behavior, and allow us to reason
with the structures of the sorting algorithms themselves. Our work only concerns the correctness
of sorting algorithms, with the goal to axiomatize sorting functions as functions satisfying
some abstract properties, independent of a given ordering, which allows us to gain
insight into how sorting relates to order and vice versa.
% and its implications on axiom of choice.
% maybe write more on its relationship to AC, in a sepreate paragraph?
