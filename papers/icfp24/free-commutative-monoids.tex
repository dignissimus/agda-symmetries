\section{Construction of Free Commutative Monoids}
\label{sec:commutative-monoids}

We now extend our constructions of free monoids to free commutative monoids.
Conceptually, adding commutativity to free monoids is "forgetting" the order
of free monoids. Finite multiset for example would be a commutative structure,
where $xs \cup ys = ys \cup xs$. Its order is "forgotten" in the sense that for example,
$\{a, a, b, c\} = \{b, a, c, a\}$. This is unlike free monoids, using $\List$ for example, 
where $[a, a, b, c] \neq [b, a, c, a]$. By formalizing free commutative monoids
we can study sorting as functions from free commutative monoids to free monoids,
which we would later demonstrate in~\cref{sec:sorting}.

\subsection{Free monoid quotiented with permutation}\label{cmon:qfreemon}
We give a general construction of free commutative monoid as a free monoid quotient.
Specific instances of this construction are given in \cref{cmon:plist} and \cref{cmon:bag}.
If $(F, \mu)$ satisfies the universal property of free monoid,
then $(F(A), i, \otimes)$ quotiented by a permutation relation $(F(A) / \approx, e, \doubleplus)$
would be the free commutative monoid on $A$.

A relation $\approx$ is a correct permutation relation iff it:
\begin{itemize}
    \item is reflexive, symmetric, transitive (equivalence),
    \item is a congruence wrt $\otimes$: $a \otimes b \to c \otimes d \to a \otimes c \approx b \otimes d$,
    \item is commutative: $a \otimes b \approx b \otimes a$, and
    \item respects $\ext{(\blank)}$: $\forall f. \, a \approx b \to \ext{f}(a) = \ext{f}(b)$.
\end{itemize}

Let $q : F(A) \to F(A) / \approx$ by the quotient map.
The generator map $\eta_A$ is given by $q \comp \mu$, and the identity element is $q(i)$.

\begin{proposition}
$(F(A) / \approx, \doubleplus, q(i))$ is a commutative monoid.
\end{proposition}

\begin{proof}
Since $\approx$ is congruence wrt $\otimes$,
we can lift $\otimes : F(A) \to F(A) \to F(A)$ to the quotient to obtain
$\doubleplus : F(A) / \approx \to F(A) / \approx \to F(A) / \approx$.
$\doubleplus$ would also satisfy the unit laws and associativity law which $\otimes$ satisfy.
Commutativity of $\doubleplus$ follows from the commutativity requirement of $\approx$,
therefore $(F(A) / \approx, \doubleplus, q(i))$ forms a commutative monoid.
\end{proof}

For clarity, we will use $(\hat{\blank})$ to denote the extension operation of $F(A)$, 
and $\ext{(\blank)}$ for the extension operation of $F(A) / \approx$.

\begin{definition}
We define the $\ext{(\blank)}$ on $f : A \to X$ to $\ext{f} : F(A) / \approx \to \mathfrak{X}$ as follow:
we first obtain $\hat{f} : F(A) \to \mathfrak{X}$ by universal property of $F$, and lift it
to $F(A) / \approx \to \mathfrak{X}$, which is allowed since $\approx$ respects $\ext{(\blank)}$.
\end{definition}

\begin{lemma}
    For all $f : A \to X$, $x : F(A)$, $\ext{f}(q(x))$ reduces to $\hat{f}(x)$.
\end{lemma}

\begin{proposition}[Universal property for $F(A) / \approx$]
    $(F(A) / \approx,\eta_A)$ is the free commutative monoid on $A$.
\end{proposition}

\begin{proof}
    To show that $\ext{(\blank)}$ is an inverse to $\blank \comp \eta_A$,
    we first show $\ext{(\blank)}$ is the right inverse to $\blank \comp \eta_A$.
    For all $f$ and $x$, $(\ext{f} \comp \eta_A)(x) = (\ext{f} \comp q)(\mu_A(x)) = \hat{f}(\mu_A(x))$.
    By universal property of $F$, $\hat{f}(\mu_A(x)) = f(x)$, therefore $(\ext{f} \comp \eta_A)(x) = f(x)$.
    By function extensionality, for any $f$, $\ext{f} \circ \eta_A = f$,
    and $(\blank \circ \eta_A) \comp \ext{(\blank)} = id$.

    To show $\ext{(\blank)}$ is the left inverse to $\blank \comp \eta_A$, we need to prove
    for any commutative monoid homomorphism $f : F(A) / \approx \to \mathfrak{X}$ and $x : F(A) / \approx$,
    $\ext{(f \comp \eta_A)}(x) = f(x)$. To prove this it is suffice to show for all $x : F(A)$,
    $\ext{(f \comp \eta_A)}(q(x)) = f(q(x))$. 
    $\ext{(f \comp \eta_A)}(q(x))$ reduces to $\widehat{(f \comp q \comp \mu_A)}(x)$.
    Note that both $f$ and $q$ are homomorphism, therefore $f \comp q$ is a homomorphism. By
    universal property of $F$ we get $\widehat{(f \comp q \comp \mu_A)}(x) = (f \comp q)(x)$,
    therefore $\ext{(f \comp \eta_A)}(q(x)) = f(q(x))$.

    We have now shown that $(\blank) \comp \eta_A$ is an equivalence from
    commutative monoid homomorphisms $F(A) / \approx \to \mathfrak{X}$
    to set functions $A \to X$, and its inverse is given by $\ext{(\blank)}$, which maps set
    functions $A \to X$ to commutative monoid homomorphisms $F(A) / \approx \to \mathfrak{X}$.
    Therefore, $F(A) / \approx$ is indeed the free commutative monoid on $A$.
\end{proof}


\subsection{List quotiented with permutation}\label{cmon:plist}
A specific instance of this construction would be $\List$ quotiented with permutation to get commutativity.
This construction is taken from \cite{joramConstructiveFinalSemantics2023}, who gave a proof on how
$\PList$ is isomorphic to free commutative monoid as a HIT. We give a direct proof of its universal
property instead.

\begin{code}
data Perm (A : UU) : List A -> List A -> UU where
  perm-refl : forall {xs} -> Perm xs xs
  perm-swap : forall {x y xs ys zs} -> Perm (xs ++ x :: y :: ys) zs
                               -> Perm (xs ++ y :: x :: ys) zs 
PList : UU -> UU
PList A = List A / Perm
\end{code}
\vspace{1em}

We have already given a proof of $\List$ being the free monoid in~\cref{mon:lists}.
By~\cref{cmon:qfreemon} it suffices to show $\Perm$ satisfies the axioms of permutation relation
to show $\PList$ is the free commutative monoid.

It is trivial to see how $\Perm$ satisfies reflexivity, symmetry, transitivity.
We can also prove $\Perm$ is congruent wrt to $\doubleplus$ by induction.
Commutativity can also be proven very similarly from the proof for $\SList$ in~\cref{slist:comm}.

\begin{theorem}\label{plist:sharp-sat}
    For all $f : A \to X$, $x, \, y : A$ and $xs, \, ys : \SList(A)$,
    $\ext{f}(xs\,\doubleplus\,x :: y :: ys) = \ext{f}(xs\,\doubleplus\,y :: x :: ys)$.
\end{theorem}

With this we can prove $\Perm$ respects $\ext{(\blank)}$.

\begin{proof}
We can prove this by induction on $xs$. For $xs = []$, by homomorphism properties of $\ext{f}$,
we can prove $\ext{f}(x :: y :: ys) = \ext{f}([ x ]) \otimes \ext{f}([ y ]) \otimes \ext{f}(ys)$.
Since the image of $\ext{f}$ is a commutative monoid, we have
$\ext{f}([ x ]) \otimes \ext{f}([ y ]) = \ext{f}([ y ]) \otimes \ext{f}([ x ])$, therefore proving
$\ext{f}(x :: y :: ys) = \ext{f}(y :: x :: ys)$. For $xs = z :: zs$, we can prove
$\ext{f}((z :: zs)\,\doubleplus\,x :: y :: ys) = \ext{f}([ z ]) \otimes \ext{f}(zs\,\doubleplus\,x :: y :: ys)$.
We can then complete the proof by induction to obtain
$\ext{f}(zs\,\doubleplus\,x :: y :: ys) = \ext{f}(zs\,\doubleplus\,y :: x :: ys)$ and reassembling
back to $\ext{f}((z :: zs)\,\doubleplus\,y :: x :: ys)$ by homomorphism properties of $\ext{f}$.
\end{proof}

\subsubsection{Representation}\label{plist:rep}
With this representation it is very easy to lift functions and properties defined on $\List$
to $\PList$ since $\PList$ is a quotient of $\List$. The inductive nature of $\PList$ makes it
very easy to define algorithms and proofs that are inductive in nature, e.g. defining insertion sort
on $\PList$ is very simple since insertion sort inductively sorts a list, which we can easily do by
pattern matching on $\PList$ since the construction of $\PList$ is definitionally inductive.
This property also makes it such that oftentimes inductively constructed $\PList$ would normalize to the
simplest form of the $\PList$, e.g. $q([ x ]) \doubleplus q([y, z])$ normalizes to $q([x,y,z])$ by
definition, saving the efforts of defining auxillary lemmas to prove their equality.

This inductive nature, however, makes it difficult to define efficient operations on $\PList$. Consider a
divide-and-conquer algorithm such as merge sort, which involves partitioning a $\PList$ of length $n+m$ into
two smaller $\PList$ of length $n$ and length $m$. The inductive nature of $\PList$ makes it such that
we must iterate all $n$ elements before we can make such a partition, thus making $\PList$ unintuitive
to work with when we want to reason with operations that involve arbitrary partitioning.

Also, whenever we define a function on $\PList$ by pattern matching we would also need to show 
the function respects $\Perm$, i.e. $\Perm as\,bs \to f(as) = f(bs)$. This can be annoying because
of the many auxiliary variables in the constructor $\term{perm-swap}$, namely $xs$, $ys$, $zs$.
We need to show $f$ would respect a swap in the list anywhere between $xs$ and $ys$, which can 
unnecessarily complicate the proof. For example in the inductive step of~\cref{plist:sharp-sat},
$\ext{f}((z :: zs)\,\doubleplus\,x :: y :: ys) = \ext{f}([ z ]) \otimes \ext{f}(zs\,\doubleplus\,x :: y :: ys)$,
to actually prove this in Cubical Agda would involve first applying associativity to prove
$(z :: zs)\,\doubleplus\,x :: y :: ys = z :: (zs\,\doubleplus\,x :: y :: ys)$, before we can actually
apply homomorphism properties of $f$. In the final reassembling step, similarly,
we also need to re-apply associativity to go from $z :: (zs\,\doubleplus\,y :: x :: ys)$
to $(z :: zs)\,\doubleplus\,y :: x :: ys$. Also since we are working with an equivalence relation we
defined ($\Perm$) and not the equality type directly, we cannot exploit the many combinators defined
in the standard library for the equality type and often needing to re-define combinators ourselves.

\subsection{Swap-List}\label{cmon:slist}
We can use HITs where adjacent swaps generate all symmetries,
for example swap-list taken from \cite{choudhuryFreeCommutativeMonoids2023}, where they have
also given a proof of its universal property.

\begin{code}
data SList (A : UU) : UU where
  nil : SList A
  _cons_ : A -> SList A -> SList A
  swap : forall x y xs -> x :: y :: xs == y :: x :: xs 
  trunc : forall x y -> (p q : x == y) -> p == q
\end{code}
\vspace{1em}

The $\term{trunc}$ constructor is necessary to truncate $\SList$ down to a set,
thereby ignoring any higher paths intrudoced by the $\term{swap}$ constructor.
This is opposed to $\List$, which does not need a $\term{trunc}$ constructor
because it does not have any path constructors, therefore it can be proven that $\List(A)$
is a set assuming $A$ is a set.

\begin{definition}[Concatenation]
    We define the concatenation operation $\doubleplus : \SList(A) \to \SList(A) \to \SList(A)$
    inductively.
    \begin{align*}
        [] \doubleplus ys & = ys \\
        (x :: xs) \doubleplus ys & = x :: (xs \doubleplus ys) \\
        \term{cong}_{\doubleplus ys}(\term{swap}(x, y, xs)) & = \term{swap}(x, y, ys \doubleplus xs)
    \end{align*}
\end{definition}

\cite{choudhuryFreeCommutativeMonoids2023} has already given a proof of $(\SList(A), \doubleplus, [])$
satisfying commutative monoidal laws. We explain the proof of $\doubleplus$ satisfying commutativity here.

\begin{lemma}[Head rearrange]\label{slist:cons}
    For all $x : A$, $xs : \SList(A)$, $x :: xs = xs \doubleplus [ x ]$.
\end{lemma}

\begin{proof}
We can prove this by induction on $xs$.
For $xs = []$ this is trivial. For $xs = y :: ys$, we have the induction hypothesis $x :: ys = ys \doubleplus [ x ]$.
By applying $y :: \blank$ on both side and then apply $\term{swap}$, we can complete the proof.
\end{proof}

\begin{theorem}[Commutativity]\label{slist:comm}
    For all $xs,\,ys : \SList(A)$, $xs \doubleplus ys = ys \doubleplus xs$.
\end{theorem}

\begin{proof}
    By induction on $xs$ we can iteratively apply~\cref{slist:cons} to move all elements of $xs$
    to after $ys$. This would move $ys$ to the head and $xs$ to the end, thereby proving
    $xs \doubleplus ys = ys \doubleplus xs$.
\end{proof}

\subsubsection{Representation}\label{slist:rep}
Much like $\PList$ and $\List$, $\SList$ is inductively defined, therefore making it very intuitive to reason
with when defining inductive operations or inductive proofs on $\SList$, however difficult to reason with
when defining operations that involve arbitrary partitioning, for reasons similar to those given
in~\cref{plist:rep}.

Unlike $\PList$ which is defined as a set quotient, this is defined as a HIT, therefore handling equalities
between $\SList$ is much simpler than $\PList$. We would still need to prove a function $f$ respects
the path constructor of $\SList$ when pattern matching, i.e. $f(x :: y :: xs) = f(y :: x :: xs)$.
Unlike $\PList$ we do not need to worry about as many auxiliary variables since swap
only happens at the head of the list, whereas with $\PList$ we would need to prove
$f(xs\,\doubleplus\,x :: y :: ys) = f(xs\,\doubleplus\,y :: x :: ys)$. One may be tempted to just remove $xs$
from the $\term{perm-swap}$ constructor such that it becomes
$\term{perm-swap} : \forall x\,y\,ys\,zs \to \Perm\,(x :: y :: ys)\,zs \to \Perm\,(y :: x :: ys)\,zs$.
However this would break $\Perm$'s congruence wrt to $\doubleplus$, therefore violating the axioms of
permutation relations. Also, since we are working with the identity type directly, properties we would
expect from $\term{swap}$, such as reflexivity, transitivity, symmetry, congruence and such all comes directly by
construction, whereas with $\Perm$ we would have to prove these properties manually.
We can also use the many combinatorics defined in the standard library for equational reasoning,
making the handling of $\SList$ equalities a lot simpler.

\subsection{Bag}\label{cmon:bag}
Alternatively, we can also quotient $\Array$ with symmetries to get commutativity.
This construction is taken from \cite{choudhuryFreeCommutativeMonoids2023}. \cite{joramConstructiveFinalSemantics2023}
also gave a similar construction, where only the index function is quotiented, instead of
the entire array. Both proved $\Bag$ is the free commutative monoid by isomorphism to
other constructions. We give a direct proof of its universal property instead.

\begin{code}
_$\approx$_ : Array A $\to$ Array A $\to$ UU
(n , f) $\approx$ (m , g) = $\Sigma$[ $\sigma$ $\in$ Iso (Fin n) (Fin m) ] v $\equiv$ w $\circ$ $\sigma$ .fun

Bag : UU -> UU
Bag A = Array A / _$\approx$_
\end{code}

Note that by the pigeonhole principle, $\sigma$ can only be constructed when $n = m$. Conceptually,
we are quotienting array by a permutation or an automorphism on the index.

We have already given a proof of $\Array$ being the free monoid in~\cref{mon:array}.
By~\cref{cmon:qfreemon} it suffices to show $\approx$ satisfies the axioms of permutation relation
to show $\Bag$ is the free commutative monoid.

\begin{proposition}
    $\approx$ is a equivalence relation.
\end{proposition}

\begin{proof}
    We can show any array $xs$ is related to itself by the identity isomorphism, therefore $\approx$ is reflexive.
    If $xs \approx ys$ by $\sigma$, we can show $ys \approx xs$ by $\sigma^{-1}$, therefore $\approx$ is symmetric.
    If $xs \approx ys$ by $\sigma$ and $ys \approx zs$ by $\phi$, we can show $xs \approx zs$ by $\sigma \comp \phi$,
    therefore $\approx$ is transitive.
\end{proof}

\begin{proposition}\label{bag:cong}
    $\approx$ is congruent wrt to $\doubleplus$.
\end{proposition}

\begin{proof}
    Given $(n, f) \approx (m, g)$ by $\sigma$ and $(u, p) \approx (v, q)$ by $\phi$,
    we want to show $(n, f) \doubleplus (u, p) \approx (m, g) \doubleplus (v, q)$ by some $\tau$.
    We can construct $\tau$ as follow:

    \begin{align*}
        \tau := \Fin[n+u] \xto{\sim} \Fin[n] + \Fin[u] \xto{\sigma,\,\phi} \Fin[m] + \Fin[v] \xto{\sim} \Fin[m+v] \\
    \end{align*}
   
    \begin{figure}[H]
        \centering
          % https://q.uiver.app/#q=WzAsMixbMCwwLCJcXHswLDEsXFxkb3RzLG4tMSwgbixuKzEsXFxkb3RzLG4rbS0xXFx9Il0sWzAsMSwiIFxce24sbisxXFxkb3RzLG4rbS0xLDAsMSxcXGRvdHMsbi0xXFx9Il0sWzAsMSwiIiwwLHsic3R5bGUiOnsidGFpbCI6eyJuYW1lIjoibWFwcyB0byJ9fX1dXQ==
    \begin{tikzcd}[ampersand replacement=\&,cramped]
    	{\{\color{red}0,1,\dots,n-1, \color{blue} n,n+1,\dots,n+u-1 \color{black}\}} \\
    	{ \{\color{red}\sigma(0),\sigma(1)\dots,\sigma(n-1), \color{blue}\phi(0),\phi(1),\dots,\phi(u-1) \color{black}\}}
        \arrow["{\sigma,\phi}", maps to, from=1-1, to=2-1]
    \end{tikzcd}
        \caption{Operation of $\tau$}
        \label{fig:enter-label}
    \end{figure}
    
\end{proof}

\begin{proposition}\label{bag:comm}
    $\approx$ is commutative.
\end{proposition}

\begin{proof}
We want to show for any arrays $(n, f)$ and $(m, g)$, $(n, f) \otimes (m, g) \approx (m, g) \otimes (n, f)$
by some $\phi$.

We can use combinators from formal operations in symmetric rig groupoids \cite{choudhurySymmetriesReversibleProgramming2022} to define $\phi$:
\begin{align*}
    \phi := \Fin[n+m] \xto{\sim} \Fin[n] + \Fin[m] \xto{\term{swap}_{+}} \Fin[m] + \Fin[n] \xto{\sim} \Fin[m+n] \\
\end{align*}

\begin{figure}[H]
    \centering
      % https://q.uiver.app/#q=WzAsMixbMCwwLCJcXHswLDEsXFxkb3RzLG4tMSwgbixuKzEsXFxkb3RzLG4rbS0xXFx9Il0sWzAsMSwiIFxce24sbisxXFxkb3RzLG4rbS0xLDAsMSxcXGRvdHMsbi0xXFx9Il0sWzAsMSwiIiwwLHsic3R5bGUiOnsidGFpbCI6eyJuYW1lIjoibWFwcyB0byJ9fX1dXQ==
\begin{tikzcd}[ampersand replacement=\&,cramped]
	{\{\color{red}0,1,\dots,n-1, \color{blue} n,n+1,\dots,n+m-1 \color{black}\}} \\
	{ \{\color{blue}n,n+1\dots,n+m-1, \color{red}0,1,\dots,n-1 \color{black}\}}
    \arrow["\phi", maps to, from=1-1, to=2-1]
\end{tikzcd}
    \caption{Operation of $\phi$}
    \label{fig:enter-label}
\end{figure}

\end{proof}

\begin{proposition}
    $\approx$ respects $\ext{(\blank)}$.
\end{proposition}

It suffices to show $\ext{f}$ is invariant under permutation: for all $\phi\colon \Fin[n]\xto{\sim}\Fin[n]$,
$\ext{f}(n, i) \id \ext{f}(n, i \circ \phi)$. To prove this we first need some lemmas.

\begin{lemma}\label{bag:tau}
Given $\phi\colon \Fin[S(n)]\xto{\sim}\Fin[S(n)]$, there is a $\tau\colon \Fin[S(n)]\xto{\sim}\Fin[S(n)]$
such that $\tau(0) = 0$, and $\ext{f}(S(n), i \circ \phi) = \ext{f}(S(n), i \circ \tau)$.
\end{lemma}

\begin{proof}
Let $k$ be $\phi^{-1}(0)$, and $k + j = S(n)$, we can construct $\tau$ as follow:
\begin{align*}
    \tau := \Fin[S(n)] \xto{\phi} \Fin[S(n)] \xto{\sim} \Fin[k+j] \xto{\sim} \Fin[k] + \Fin[j]
    \xto{\term{swap}_{+}} \Fin[j] + \Fin[k] \xto{\sim} \Fin[j+k] \xto{\sim} \Fin[S(n)]
\end{align*}

\begin{figure}[H]
    \centering
    \begin{tikzcd}[ampersand replacement=\&,cramped]
    	{\{\color{blue}0, 1, 2, \dots, \color{red}k, k+1, k+2, \dots \color{black}\}} \\
    	{ \{\color{blue}x, y, z, \dots, \color{red}0, u, v, \dots \color{black}\}}
        \arrow["\phi", maps to, from=1-1, to=2-1]
    \end{tikzcd}
    \hspace{1em}
    \begin{tikzcd}[ampersand replacement=\&,cramped]
    	{\{\color{blue}0, 1, 2, \dots, \color{red}k, k+1, k+2, \dots \color{black}\}} \\
    	{ \{\color{red}0, u, v, \dots, \color{blue}x, y, z, \dots \color{black}\}}
        \arrow["\tau", maps to, from=1-1, to=2-1]
    \end{tikzcd}
    \caption{Operation of $\tau$ constructed from $\phi$}
    \label{fig:enter-label}
\end{figure}

It is trivial to prove $\ext{f}(S(n), i \circ \phi) = \ext{f}(S(n), i \circ \tau)$ since the only
operation on indices in $\tau$ is $\term{swap}_{+}$. It suffices to show $(S(n), i \circ \phi)$
can be decomposed into two arrays such that $(S(n), i \circ \phi) = (k, g) \doubleplus (j, h)$
for some $g$ and $h$. Since the image of $\ext{f}$ is a commutative monoid and $\ext{f}$ is a homomorphism,
$\ext{f}((k, g) \doubleplus (j, h)) = \ext{f}(k, g) \otimes \ext{f}(j, h) = \ext{f}(j, h) \otimes \ext{f}(k, g) =
\ext{f}((j, h) \doubleplus (k, g))$, thereby proving $\ext{f}(S(n), i \circ \phi) = \ext{f}(S(n), i \circ \tau)$.

\end{proof}


\begin{lemma}\label{bag:punch}
Given $\tau\colon \Fin[S(n)]\xto{\sim}\Fin[S(n)]$ where $\tau(0) = 0$,
there is a $\psi : \Fin[n] \xto{\sim} \Fin[n]$ such that $\tau \circ S = S \circ \psi$.
\end{lemma}

\begin{proof}
We can construct $\psi$ by $\psi(x) = \tau(S(x)) - 1$.
Since $\tau$ maps only 0 to 0 by assumption, $\forall x. \, \tau(S(x)) > 0$, therefore
the $(- 1)$ is well defined. This can be thought of as a special case $k = 0$ of the punch-in and punch-out
equivalence used to construct Lehmer codes in~\cite{choudhurySymmetriesReversibleProgramming2022}.

\begin{figure}[H]
    \centering
    \begin{tikzcd}[ampersand replacement=\&,cramped]
    	{\{\color{blue}0, \color{red}1, 2, 3, \dots \color{black}\}} \\
    	{ \{\color{blue}0, \color{red} x, y, z \dots \color{black}\}}
        \arrow["\tau", maps to, from=1-1, to=2-1]
    \end{tikzcd}
    \hspace{1em}
    \begin{tikzcd}[ampersand replacement=\&,cramped]
    	{\{\color{red}0, 1, 2, \dots \color{black}\}} \\
    	{ \{\color{red} x-1, y-1, z-1 \dots \color{black}\}}
        \arrow["\psi", maps to, from=1-1, to=2-1]
    \end{tikzcd}
    \caption{Operation of $\psi$ constructed from $\tau$}
    \label{fig:enter-label}
\end{figure}
\end{proof}


We now prove our main theorem.

\begin{theorem}[Permutation invariance]\label{bag:perm-sat}
For all $\phi\colon \Fin[n]\xto{\sim}\Fin[n]$, $\ext{f}(n, i) \id \ext{f}(n, i \circ \phi)$.
\end{theorem}

\begin{proof}
We can prove this by induction on $n$.
\begin{itemize}
    \item On $n = 0$, $\ext{f}(0, i) \id \ext{f}(0, i \circ \phi) = e$.
    \item On $n = S(m)$,
        \begin{align*}
        & \ext{f}(S(m), i \circ \phi) \\
        & = \ext{f}(S(m), i \circ \tau) & \text{by~\cref{bag:tau}} \\
        & = f(i(\tau(0))) \otimes \ext{f}(m, i \circ \tau \circ S) & \text{by definition of $\ext{(\blank)}$} \\
        & = f(i(0)) \otimes \ext{f}(m, i \circ \tau \circ S) & \text{by construction of $\tau$} \\
        & = f(i(0)) \otimes \ext{f}(m, i \circ S \circ \psi) &\text{by~\cref{bag:punch}} \\
        & = f(i(0)) \otimes \ext{f}(m, i \circ S) & \text{induction} \\
        & = \ext{f}(S(m), i) & \text{by definition of $\ext{(\blank)}$}
        \end{align*}
\end{itemize}
\end{proof}

\subsubsection{Representation}\label{bag:rep}
Unlike $\PList$ and $\SList$, $\Bag$ and its underlying construction $\Array$ are not inductively defined,
making it difficult to work with when trying to do induction on them. For example,
in the proof~\cref{array:univ}, two lemmas~\cref{array:eta-suc} and~\cref{array:split} are needed to do
induction on $\Array$, as opposed to $\List$ and its quotients, where we can do induction simply by
pattern matching. Much like $\PList$, when defining functions on $\Bag$, we need to show they respect
$\approx$, i.e. $as \approx bs \to f(as) = f(bs)$. This is notably much more difficult than
$\PList$ and $\SList$, because whereas with $\PList$ and $\SList$ we only need to consider "small permutations"
(i.e. swapping adjacent elements), with $\Bag$ we need to consider all possible permutations. For example,
in the proof of~\cref{bag:perm-sat}, we need to first construct a $\tau$ which satisfies $\tau(0) = 0$ and prove
$\ext{f}(n, i \comp \sigma) = \ext{f}(n, i \comp \tau)$ before we can apply induction.

Also, on a more technical note, since $\Array$ and $\Bag$ are not simple data types, the definition of
the monoid operation on them $\doubleplus$ are necessarily more complicated, and unlike $\List$, $\PList$
and $\SList$, constructions of $\Array$ and $\Bag$ via $\doubleplus$ often would not normalize into a
very simple form, but would instead expand into giant trees of terms. This makes it such that when working
with $\Array$ and $\Bag$ we need to be very careful or otherwise Agda would be stuck trying to display
the normalized form of $\Array$ and $\Bag$ in the goal and context menu. Type-checking also becomes a lengthy
process that tests if the user possesses the virtue of patience.

However, performing arbitrary partitioning with $\Array$ and $\Bag$ is much easier than
$\List$, $\SList$, $\PList$. For example,
one can simply use the combinator $\Fin[n+m] \xto{\sim} \Fin[n] + \Fin[m]$ to partition the array,
then perform operations on the partitions such as swapping in~\cref{bag:comm},
or perform operations on the partitions individually such as two individual permutation in~\cref{bag:cong}.
This makes it such that when defining divide-and-conquer algorithms such as merge sort,
$\Bag$ and $\Array$ are more natural to work with than $\List$, $\SList$, and $\PList$.
