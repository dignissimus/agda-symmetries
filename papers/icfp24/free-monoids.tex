\section{Construction of Free Monoids}
\label{sec:monoids}

Free monoids are well-known to functional programmers. In this section, we consider various constructions of them in type theory, and prove their universal property. Because we are working in univalent type theory, these constructions are equivalent as types (and as monoids).

\subsection{List}

The simplest inductive data structure is a cons-list, which is a free monoid~\cite{dubucFreeMonoids1974}. As a type, this is given by:

\begin{definition}{Lists}
The type of lists on a type $A$ is given by:
\vspace{1em}
\begin{code}
data List (A : UU) : UU where
  nil : List A
  _cons_ : A -> List A -> List A
\end{code}
\vspace{1em}
\end{definition}

An example of a list would be $3 :: 1 :: 2 :: []$, alternatively denoted as $[3, 1, 2]$ for convenience.

The generators map is the singleton: $\eta_A(a) =$~[ a ]~$= a :: []$, 
the identity element is the empty list $[]$,
and the monoid operation $\doubleplus$ is given by the concatenation operation.

\begin{definition}[Concatenation]
We define the concatenation operation $\doubleplus : List(A) \to List(A) \to List(A)$ inductively.
\begin{align*}
    [] \doubleplus ys & = ys \\
    (x :: xs) \doubleplus ys & = x :: (xs \doubleplus ys)
\end{align*}
\end{definition}

We leave the proof that concatenation satisfies the unit laws and associativity law of monoids to the readers.

\begin{definition}[Extension/fold]
We define the $\ext{(\blank)}$ on $f : A \to X$ as:
\begin{align*}
    \ext{f}([]) & = e \\
    \ext{f}(x :: xs) & =  f(x) \otimes \ext{f}(xs)
\end{align*} 
\end{definition}

\begin{proposition}
    $\ext{(\blank)}$ lifts a function $f : A \to X$ to a monoid homomorphism $\ext{f} : List(A) \to \mathfrak{X}$.
\end{proposition}

\begin{proof}
    To show that $\ext{f}$ is a monoid homomorphism,
    we need to show $\ext{f}(xs \doubleplus ys) = \ext{f}(xs) \otimes \ext{f}(ys)$.
    We can do so by induction on $xs$.

    Case []:
    $\ext{f}([] \doubleplus ys) = \ext{f}(ys)$,
    and $\ext{f}([]) \otimes \ext{f}(ys) = e \otimes \ext{f}(ys) = \ext{f}(ys)$
    by definition of $\ext{(\blank)}$. Therefore, we have
    $\ext{f}([] \doubleplus ys) = \ext{f}([]) \otimes \ext{f}(ys)$.

    Case $x :: xs$:
    \begin{align*}
        & \ext{f}((x :: xs) \doubleplus ys) \\
        & = \ext{f}(([ x ] \doubleplus xs) \doubleplus ys) & \text{by definition of concatenation} \\
        & = \ext{f}([ x ] \doubleplus (xs \doubleplus ys)) & \text{by associativity} \\
        & = \ext{f}(x :: (xs \doubleplus ys)) & \text{by definition of concatenation} \\
        & = f(x) \otimes \ext{f}(xs \doubleplus ys) & \text{by definition of $\ext{(\blank)}$} \\
        & = f(x) \otimes (\ext{f}(xs) \otimes \ext{f}(ys)) & \text{by induction} \\
        & = (f(x) \otimes \ext{f}(xs)) \otimes \ext{f}(ys) & \text{by associativity} \\
        & = \ext{f}(x :: xs) \otimes \ext{f}(ys) & \text{by definition of $\ext{(\blank)}$} \\
    \end{align*}

    Therefore, $\ext{(\blank)}$ does correctly lift a function to a monoid homomorphism.
\end{proof}

\begin{proposition}[Universal property for List]
    $(List(A),\eta_A)$ is the free monoid on $A$.
\end{proposition}

\begin{proof}
    To show that $\ext{(\blank)}$ is an inverse to $\blank \comp \eta_A$,
    we first show $\ext{(\blank)}$ is the right inverse to $\blank \comp \eta_A$.
    For all $f$ and $x$, $(\ext{f} \circ \eta_A)(x) = \ext{f}(x :: []) = f(x) \otimes e = f(x)$,
    therefore by function extensionality, for any $f$, $\ext{f} \circ \eta_A = f$,
    and $(\blank \circ \eta_A) \comp \ext{(\blank)} = id$.
    
    To show $\ext{(\blank)}$ is the left inverse to $\blank \comp \eta_A$, we need to prove
    for any monoid homomorphism $f : List(A) \to \mathfrak{X}$, $\ext{(f \comp \eta_A)}(xs) = f(xs)$.
    We can do so by induction on $xs$.

    Case []: $\ext{(f \comp \eta_A)}([]) = e$ by definition of the $\ext{(\blank)}$ operation,
    and $f([]) = e$ by homomorphism properties of $f$. Therefore, $\ext{(f \comp \eta_A)}([]) = f([])$.

    Case $x :: xs$: 
    \begin{align*}
        & \ext{(f \comp \eta_A)}(x :: xs) \\
        & = (f \comp \eta_A)(x) \otimes \ext{(f \comp \eta_A)}(xs) & \text{by definition of $\ext{(\blank)}$} \\
        & = (f \comp \eta_A)(x) \otimes f(xs) & \text{by induction} \\
        & = f([x]) \otimes f(xs) & \text{by definition of $\eta_A$} \\
        & = f([x] \doubleplus xs) & \text{by homomorphism properties of $f$} \\
        & = f(x :: xs) & \text{by definition of concatenation}
    \end{align*}

    By function extensionality, $\ext{(\blank)} \comp (\blank \circ \eta_A) = id$.
    Therefore, $\ext{(\blank)}$ and $(\blank) \circ [\_]$ are inverse of each other.

    We have now shown that $(\blank) \comp \eta_A$ is an equivalence from monoid homomorphisms $List(A) \to \mathfrak{X}$
    to set functions $A \to X$, and its inverse is given by $\ext{(\blank)}$, which maps set
    functions $A \to X$ to monoid homomorphisms $List(A) \to \mathfrak{X}$. Therefore, lists are indeed
    free monoids.
    
\end{proof}
\subsection{Array}

Alternatively, we can construct free monoid as a pair of natural number $n$, denoting the length of the array,
and a lookup function $\Fin[n] \to A$, mapping each index to an element of $A$~\cite{dubucFreeMonoids1974}.

\begin{definition}{Arrays}
The types of arrays on a type $A$ and $\Fin[n]$ are given by:
\vspace{1em}
\begin{code}
Fin : $\mathbb{N}$ -> UU
Fin n = $\Sigma$[ m $\in$ $\mathbb{N}$ ] (m < n)

Array : UU -> UU
Array A = $\Sigma$[ n $\in$ $\mathbb{N}$ ] (Fin n $\to$ A)
\end{code}
\vspace{1em}
\end{definition}

An example of an array would be $(3, \lambda\{ 0 \mapsto 3, 1 \mapsto 1, 2 \mapsto 2 \})$, which would
be equivalent to the list $[3, 1, 2]$.

The generators map is the singleton: $\eta_A(a) = (1, \lambda\{ 0 \mapsto a \})$, 
the identity element is $(0, \lambda\{\})$
and the monoid operation $\doubleplus$ is given by the concatenation operation.

\begin{lemma}[Array identity]\label{array:zero-is-id}
Any zero-length array $(0, f)$ is equal to the identity element.
\end{lemma}

\begin{proof}
We need to show $f : \Fin[0] \to A$ is equal to $\lambda\{\}$. By function extensionality we can prove
this by showing for all $x : \emptyt$, $f(x) = (\lambda\{\})(x)$, which we can prove by absurdity elimination on $x$.
Therefore, any array $(0, f)$ is equal to $(0, \lambda\{\})$.
\end{proof}

\begin{definition}[Concatenation]
We define the concatenation operation $\doubleplus : Array(A) \to Array(A) \to Array(A)$,
and the combine operation $\oplus : (\Fin[n] \to A) \to (\Fin[m] \to A) \to (\Fin[n+m] \to\nolinebreak A)$.
\begin{align*}
    (n , f) \doubleplus (m , g) & = (n + m , f \oplus g) \\
    (f \oplus g)(k) & = \begin{cases}
      f(k) & \text{if}\ k < n \\
      g(k - n) & \text{otherwise}
    \end{cases}
\end{align*} 
\end{definition}

\begin{proposition}[Left unit]\label{array:unitl}
$(Array(A), \doubleplus)$ satisfies left unit law.
\end{proposition}

\begin{proof}
We want to show $(0, \lambda\{\}) \doubleplus (n, f) = (n, f)$.
\begin{align*}
    (0 , \lambda\{\}) \doubleplus (n , f) & = (0 + n , \lambda\{\} \oplus f) \\
    (\lambda\{\} \oplus f)(k) & = \begin{cases}
      (\lambda\{\})(k) & \text{if}\ k < 0 \\
      f(k - 0) & \text{otherwise}
    \end{cases}
\end{align*} 

It is trivial to see the length matches: $0 + n = n$. We also need to show $\lambda\{\} \oplus f = f$.
Since $n < 0$ for any $n : \mathbb{N}$ is impossible, $(\lambda\{\} \oplus f)(k)$ would always reduce to
$f(k - 0) = f(k)$, therefore $(0, \lambda\{\}) \doubleplus (n, f) = (n, f)$.
\end{proof}

\begin{proposition}[Right unit]
$(Array(A), \doubleplus)$ satisfies right unit law.
\end{proposition}

\begin{proof}
We want to show $(n, f) \doubleplus (0, \lambda\{\}) = (n, f)$.
\begin{align*}
    (n, f) \doubleplus (0 , \lambda\{\}) & = (n + 0, f \oplus \lambda\{\}) \\
    (f \oplus \lambda\{\})(k) & = \begin{cases}
      f(k) & \text{if}\ k < n \\
      (\lambda\{\})(k - 0) & \text{otherwise}
    \end{cases}
\end{align*} 

It is trivial to see the length matches: $n + 0 = n$. We also need to show $f \oplus \lambda\{\} = f$.
We note that the type of $f \oplus \lambda\{\}$ is $\Fin[n + 0] \to A$, therefore $k$ is of the type $\Fin[n + 0]$.
Since $\Fin[n+0] \cong \Fin[n]$, it must always hold that $k < n$, therefore $(f \oplus \lambda\{\})(k)$ must
always reduce to $f(k)$. Thus, $(n, f) \doubleplus (0, \lambda\{\}) = (n, f)$.
\end{proof}


\begin{proposition}[Associativity]
$(Array(A), \doubleplus)$ satisfies associativity law.
\end{proposition}

\begin{proof}
We want to show for any array $(n, f)$, $(m, g)$, $(o, h)$,
$((n, f) \doubleplus (m, g)) \doubleplus (o, h) = (n, f) \doubleplus ((m, g) \doubleplus (o, h))$.

\begin{align*}
    ((n, f) \doubleplus (m, g)) \doubleplus (o, h) & = ((n + m) + o, (f \oplus g) \oplus h) \\
    ((f \oplus g) \oplus h)(k) & = \begin{cases}
        \begin{cases}
            f(k) & \text{if}\ k < n \\
            g(k - n) & \text{otherwise}
        \end{cases}
      & \text{if}\ k < n + m \\
      h(k - (n + m)) & \text{otherwise}
    \end{cases} \\
    (n, f) \doubleplus ((m, g) \doubleplus (o, h)) & = (n + (m + o), f \oplus (g \oplus h)) \\
    (f \oplus (g \oplus h))(k) & = \begin{cases}
        f(k) &\ \text{k < n} \\
        \begin{cases}
            g(k - n) & \text{k - n < m} \\
            h(k - n - m) & \text{otherwise} \\
        \end{cases} & \text{otherwise}
    \end{cases}
\end{align*} 

We first case split on $k < n + m$ then $k < n$.

Case $k < n + m$, $k < n$: Both $(f \oplus (g \oplus h))(k)$ and $((f \oplus g) \oplus h)(k)$ reduce to $f(k)$.

Case $k < n + m$, $k \geq n$: $((f \oplus g) \oplus h)(k)$ reduce to $g(k - n)$ by definition.
To show $(f \oplus (g \oplus h))(k)$ would also reduce to $g(k - n)$, we first need to show $\neg(k < n)$,
which follows from $k \geq n$. We then need to show $k - n < m$.
This can be done by simply subtracting $n$ from both side on $k < n + m$, which is well defined since $k \geq n$. 

Case $k \geq n + m$: $((f \oplus g) \oplus h)(k)$ reduce to $h(k - (n + m))$ by definition.
To show $(f \oplus (g \oplus h))(k)$ would also reduce to $h(k - (n + m))$,
we first need to show $\neg(k < n)$, which follows from $k \geq n + m$.
We then need to show $\neg(k - n < m)$, which also follows from $k \geq n + m$.
We now have $(f \oplus (g \oplus h))(k) = h(k - n - m)$. Since $k \geq n + m$, $h(k - n - m)$ is well defined
and is equal to $h(k - (n + m))$, therefore $(f \oplus (g \oplus h))(k) = (f \oplus g) \oplus h)(k) = h(k - (n + m))$.

In all cases $(f \oplus (g \oplus h))(k) = ((f \oplus g) \oplus h)(k)$, therefore associativity holds.
\end{proof}

\begin{lemma}[Array cons]\label{array:eta-suc}
Any array $(S(n), f)$ is equal to $\eta_A (f(0)) \doubleplus (n, f \comp S)$.
\end{lemma}

\begin{proof}
We want to show $\eta_A (f(0)) \doubleplus (n, f \comp S) = (S(n), f)$.
\begin{align*}
    (1, \lambda\{ 0 \mapsto f(0) \}) \doubleplus (n , f \comp S) &
    = (1 + n, \lambda\{ 0 \mapsto f(0) \} \oplus (f \comp S)) \\
    (\lambda\{ 0 \mapsto f(0) \} \oplus (f \comp S))(k) & = \begin{cases}
      f(0) & \text{if}\ k < 1 \\
      (f \comp S)(k - 1) & \text{otherwise}
    \end{cases}
\end{align*} 

It is trivial to see the length matches: $1 + n = S(n)$. We need to show 
$(\lambda\{ 0 \mapsto f(0) \} \oplus (f \comp S)) = f$.
We prove by case splitting on $k < 1$.
On $k < 1$, $(\lambda\{ 0 \mapsto f(0) \} \oplus (f \comp S))(k)$ reduces to $f(0)$.
Since, the only possible for $k$ when $k < 1$ is 0, $(\lambda\{ 0 \mapsto f(0) \} \oplus (f \comp S))(k) = f(k)$
when $k < 1$.
On $k \geq 1$, $(\lambda\{ 0 \mapsto f(0) \} \oplus (f \comp S))(k)$ reduces to $(f \comp S)(k - 1) = f(S(k - 1))$.
Since $k \geq 1$, $S(k - 1) = k$, therefore $(\lambda\{ 0 \mapsto f(0) \} \oplus (f \comp S))(k) = f(k)$
when $k \geq 1$.
Thus, in both cases, $(\lambda\{ 0 \mapsto f(0) \} \oplus (f \comp S)) = f$.
\end{proof}

\begin{lemma}[Array split]\label{array:split}
For any array $(S(n), f)$ and $(m, g)$, $(n + m, (f \oplus g) \comp S) = (n, f \comp S) \doubleplus (m, g)$.
\end{lemma}

Informally, this means given an non-empty array $xs$ and any array $ys$,
concatenating $xs$ with $ys$ then dropping the first element is the same as
dropping the first element of $xs$ then concatenating with $ys$.

\begin{proof}
It is trivial to see both array have length $n + m$. We want to show $(f \oplus g) \comp S = (f \comp S) \oplus g$.
\begin{align*}
   ((f \oplus g) \comp S)(k) & = \begin{cases}
       f(S(k)) & \text{if}\ S(k) < S(n) \\
       g(S(k) - S(n)) & \text{otherwise}
   \end{cases} \\
   ((f \comp S) \oplus g)(k) & = \begin{cases}
       (f \comp S)(k) & \text{if}\ k < n \\
       g(k - n) & \text{otherwise}
   \end{cases}
\end{align*}

We prove by case splitting on $k < n$.
On $k < n$, $((f \oplus g) \comp S)(k)$ reduces to $f(S(k))$ since $k < n$ implies $S(k) < S(n)$,
and $((f \comp S) \oplus g)(k)$ reduces to $(f \comp S)(k)$ by definition, therefore they are equal.
On $k \geq n$, $((f \oplus g) \comp S)(k)$ reduces to $g(S(k) - S(n)) = g(k - n)$,
and $((f \comp S) \oplus g)(k)$ reduces to $g(k - n)$ by definition, therefore they are equal.
\end{proof}

\begin{definition}
We define the $\ext{(\blank)}$ on $f : A \to X$:
\begin{align*}
    \ext{f}(0 , g) & = e \\
    \ext{f}(S(n) , g) & = f(g(0)) \otimes \ext{f}(n , g \circ S)
\end{align*} 
\end{definition}

\begin{proposition}
    $\ext{(\blank)}$ lifts a function $f : A \to X$ to a monoid homomorphism $\ext{f} : Array(A) \to \mathfrak{X}$.
\end{proposition}

\begin{proof}
    To show that $\ext{f}$ is a monoid homomorphism,
    we need to show $\ext{f}(xs \doubleplus ys) = \ext{f}(xs) \otimes \ext{f}(ys)$.
    We can do so by induction on $xs$.

    Case $(0, g)$:
    We have $g = \lambda\{\}$ by~\ref{array:zero-is-id}.
    $\ext{f}((0, \lambda\{\}) \doubleplus ys) = \ext{f}(ys)$ by ~\ref{array:unitl},
    and $\ext{f}(0, \lambda\{\}) \otimes \ext{f}(ys) = e \otimes \ext{f}(ys) = \ext{f}(ys)$
    by definition of $\ext{(\blank)}$. Therefore,
    $\ext{f}((0, \lambda\{\}) \doubleplus ys) = \ext{f}(0, \lambda\{\}) \otimes \ext{f}(ys)$.

    Case $(S(n), g)$: Let $ys$ be $(m, h)$.
    \begin{align*}
       & \ext{f}((S(n), g) \doubleplus (m, h)) \\
       & = \ext{f}(S(n + m), g \oplus h) & \text{by definition of concatenation} \\
       & = f((g \oplus h)(0)) \otimes \ext{f}(n + m, (g \oplus h) \comp S) & \text{by definition of $\ext{(\blank)}$} \\
       & = f(g(0)) \otimes \ext{f}(n + m, (g \oplus h) \comp S) & \text{by definition of $\oplus$, and $0 < S(n)$} \\
       & = f(g(0)) \otimes \ext{f}((n, g \comp S) \doubleplus (m, h)) & \text{by~\ref{array:split}} \\
       & = f(g(0)) \otimes (\ext{f}(n, g \comp S) \otimes \ext{f}(m, h))) & \text{by induction} \\
       & = (f(g(0)) \otimes \ext{f}(n, g \comp S)) \otimes \ext{f}(m, h)) & \text{by associativity} \\
       & = \ext{f}(S(n), g) \otimes \ext{f}(m, h)) & \text{by definition of $\ext{(\blank)}$} \\
    \end{align*}

    Therefore, $\ext{(\blank)}$ does correctly lift a function to a monoid homomorphism.
\end{proof}

\begin{proposition}[Universal property for Array]
    $(Array(A),\eta_A)$ is the free monoid on $A$.
\end{proposition}

\begin{proof}
    To show that $\ext{(\blank)}$ is an inverse to $\blank \comp \eta_A$,
    we first show $\ext{(\blank)}$ is the right inverse to $\blank \comp \eta_A$.
    For all $f$ and $x$, $(\ext{f} \circ \eta_A)(x) = \ext{f}(1, \lambda\{0 \mapsto x\}) = f(x) \otimes e = f(x)$,
    therefore by function extensionality, for any $f$, $\ext{f} \circ \eta_A = f$,
    and $(\blank \circ \eta_A) \comp \ext{(\blank)} = id$.
    
    To show $\ext{(\blank)}$ is the left inverse to $\blank \comp \eta_A$, we need to prove
    for any monoid homomorphism $f : Array(A) \to \mathfrak{X}$, $\ext{(f \comp \eta_A)}(xs) = f(xs)$.
    We can do so by induction on $xs$.

    Case $(0, g)$:
    By~\ref{array:zero-is-id} we have $g = \lambda\{\}$.
    $\ext{(f \comp \eta_A)}(0, \lambda\{\}) = e$ by definition of the $\ext{(\blank)}$ operation,
    and $f(0, \lambda\{\}) = e$ by homomorphism properties of $f$.
    Therefore, $\ext{(f \comp \eta_A)}(0, g) = f(0, g)$.

    Case $(S(n), g)$, we prove it in reverse: 
    \begin{align*}
        & f(S(n), g) \\
        & = f(\eta_A(g(0)) \doubleplus (n, g \comp S)) & \text{by~\ref{array:eta-suc}} \\
        & = f(\eta_A(g(0))) \otimes f(n, g \comp S) & \text{by homomorphism properties of $f$} \\
        & = (f \comp \eta_A)(g(0)) \otimes \ext{(f \comp \eta_A)}(n, g \comp S) & \text{by induction} \\
        & = \ext{(f \comp \eta_A)}(S(n), g) & \text{by definition of $\ext{(\blank)}$} 
    \end{align*}

    By function extensionality, $\ext{(\blank)} \comp (\blank \circ \eta_A) = id$.
    Therefore, $\ext{(\blank)}$ and $(\blank) \circ [\_]$ are inverse of each other.

    We have now shown that $(\blank) \comp \eta_A$ is an equivalence from monoid homomorphisms $List(A) \to \mathfrak{X}$
    to set functions $A \to X$, and its inverse is given by $\ext{(\blank)}$, which maps set
    functions $A \to X$ to monoid homomorphisms $Array(A) \to \mathfrak{X}$. Therefore, arrays are indeed
    free monoids.
    
\end{proof}

