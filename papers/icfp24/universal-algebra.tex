\section{Universal Algebra}
\label{sec:universal-algebra}

We first develop some basic notions from universal algebra and equational
logic~\cite{birkhoffStructureAbstractAlgebras1935}.
%
Universal algebra is the abstract study of algebraic structures, which have (algebraic) operations and (universal)
equations.
%
This gives us some vocabulary and a framework to express our results in.
%
The point of view we take is the standard category-theoretic approach to universal algebra, which predates the Lawvere
theory or abstract clone point of view.
%
We keep a running example of monoids in mind, while explaining and defining the abstract concepts.
% In the language of universal algebra, such a structure is formalized by giving a signature of operations, and a
% structure being a carrier set with functions that interpret these operations. We describe such a framework for universal
% algebra in HoTT, as follows.

\subsection{Algebras}
\label{sec:universal-algebra:algebras}

\begin{definition}[Signature]
    \label{def:signature}
    A signature, denoted $\sig$, is a (dependent) pair consisting of:
    \begin{itemize}
        \item a set of operations, $\op : \Set$,
        \item an arity function for each symbol, $\ar : \op \to \Set$.
    \end{itemize}
\end{definition}

\begin{example}
    A monoid is a set with an identity element (or a nullary operation), and a binary multiplication operation.
    %
    The signature for monoids $\sigMon$ is encoded as:
    $(\Fin[2],\lambda \{0 \mapsto \Fin[0] ; 1 \mapsto \Fin[2] \})$.
    %
    Informally, the set of operations is the two-element set $\{e,\mult\}$, which is written as $\Fin[2]$,
    and the arity function picks out a (finite) set denoting the (finite) arity of each operation.
    %
    Of course, this is an example of a finitary signature,
    but in general the arity function can be any (not necessarily finite) set.
\end{example}

Every signature $\sig$ induces a signature functor on $\Set$, written $\Sig$.

\begin{definition}[Signature functor]
    \label{def:signature-functor}
    \begin{align*}
        \Sig \colon \Set & \to \Set                         \\
        X                & \mapsto \dsum{o:\op}{X^{\ar(o)}} \\
        X \xto{f} Y      & \mapsto
        \dsum{o:\op}{X^{\ar(o)}}
        \xto{(o, \blank \comp f)}
        \dsum{o:\op}{Y^{\ar(o)}}
    \end{align*}
\end{definition}

\begin{example}
    The signature functor for monoids, $\SigMon$, assigns to a carrier set $X$,
    the sets of inputs for each operation.
    %
    Expanding the dependent product on $\Fin[2]$, we obtain a coproduct or disjoint union of sets:
    $\SigMon(X) \eqv X^{\Fin[0]} + X^{\Fin[2]} \eqv \unitt + (X \times X)$.
\end{example}

A $\sig$-structure is given by a carrier set, with functions interpreting each operation symbol.
%
The signature functor applied to a carrier set gives the inputs to each operation, and the output is simply a map back
to the carrier set.
%
Formally, these two pieces of data are an algebra for the $\Sig$ functor.
%
We write $\str{X}$ for a $\sig$-structure with carrier set $X$, following the model-theoretic notational convention.

\begin{definition}[Structure]
    A $\sig$-structure $\str{X}$ is an $\Sig$-algebra, that is, a pair consisting of:
    \begin{itemize}
        \item a carrier set $X$, and
        \item an algebra map $\alpha_{X}\colon \Sig(X) \to X$.
    \end{itemize}
\end{definition}

\begin{example}
    Concretely, an $\SigMon$-algebra has the type
    \[
        \alpha_{X} : \SigMon(X) \to X \eqv \unitt + (X \times X) \to X \eqv (\unitt \to X) \times (X \times X \to X)
    \]
    which is the pair of functions interpreting the two operations.
    The natural numbers with the carrier set $\Nat$,
    and the constant $0$ and the addition operation $+$ give an example of such a structure.
    %
    Similarly, with the constant $1$ and the multiplication operation $\times$, we have another monoid structure.
\end{example}

One of the advantages of an abstract framework is that we can readily leverage existing abstract machinery.
%
The notion of a homomorphism between two $\sig$-structures is a standard concept --
in the categorical framework, this is simply a map between the two $\Sig$-algebras.

\begin{definition}[Homomorphism]
    A homomorphism between two $\sig$-structures $\str{X}$ and $\str{Y}$ is a morphism of $\Sig$-algebras,
    that is, a map $f : X \to Y$ such that the following diagram commutes:
    \[
        \begin{tikzcd}
            \Sig(X) \arrow[r, "\alpha_{X}"] \arrow[d, "\Sig(f)"']
            & X \arrow[d, "f"] \\
            \Sig(Y) \arrow[r, "\alpha_{Y}"']
            & Y
        \end{tikzcd}
    \]
\end{definition}

\begin{example}
    To see concretely that this definition captures the notion of a homomorphism between two monoids ($\Sig$-algebras)
    $\str{X}$ and $\str{Y}$, we note that the top half of the diagram produces the map:
    $\unitt + (X \times X) \xto{\alpha_{X}} X \xto{f} Y$, which applies $f$ to the output of each operation,
    and the bottom half of the diagram produces the map:
    $\unitt + (X \times X) \xto{\SigMon(f)} \unitt + (Y \times Y) \xto{\alpha_{Y}} Y$,
    and the commutativity of the diagram says that these two maps are equal.
    %
    In other words, a homomorphism between $X$ and $Y$ is a map $f$ on the carrier sets that commutes with the
    interpretation of the monoid operations, or simply, preserves the monoid structure.
\end{example}

For a fixed signature $\sig$,
the category of $\Sig$-algebras and their morphisms form a category of algebras,
written $\SigAlg$, or simply, $\sigAlg$,
given by the obvious definitions of identity and composition of the underlying functions.

\subsection{Free Algebras}
\label{sec:universal-algebra:free-algebras}

The category $\sigAlg$ is a category of structured sets and structure-preserving maps,
which is an example of a concrete category, that admits a forgetful functor to $U : \sigAlg \to \Set$,
which forgets the structure and retains only the carrier set.
%
Note that in our notation, $U(\str{X})$ is simply $X$, a fact that we can exploit to simplify our notation.
%
The natural question to ask is whether this forgetful functor has a left adjoint, and further, if it is monadic.

This is the construction of the free $\sig$-algebra on a set, which is a standard construction in universal algebra,
known as the term algebra (or the absolutely free algebra without equations).
%
Rather than defining the left adjoint directly, we rephrase the universal property of free algebras in more concrete
terms.

\begin{definition}[Free Algebras]
    \label{def:free-algebras}
    A free $\sig$-algebra construction consists of the following data:
    \begin{itemize}
        \item a set $F(X)$, for every set $X$,
        \item a $\sig$-structure on $F(X)$, written as $\str{F}(X)$,
        \item a universal map $\eta_X : X \to F(X)$, for every $X$, such that,
        \item for any $\sig$-algebra $\str{Y}$, the operation
              assigning to each homomorphism $f : \str{X} \to \str{Y}$,
              the function $f \comp \eta_X : X \to Y$ (or, post-composition with $\eta_X$),
              is an equivalence.
    \end{itemize}
    \[
        \begin{tikzcd}[ampersand replacement=\&]
            \str{F}(X) \\
            \\
            \str{Y}
            \arrow["f"', dotted, from=1-1, to=3-1]
        \end{tikzcd}
        \mapsto
        % https://q.uiver.app/#q=WzAsMyxbMCwwLCJYIl0sWzIsMCwiRihYKSJdLFsyLDIsIlkiXSxbMCwxLCJcXGV0YV97WH0iLDAseyJjb2xvdXIiOlswLDYwLDYwXX0sWzAsNjAsNjAsMV1dLFswLDIsImYgXFxjb21wIFxcZXRhX3tYfSIsMl0sWzEsMiwiZiIsMCx7InN0eWxlIjp7ImJvZHkiOnsibmFtZSI6ImRvdHRlZCJ9fX1dXQ==
        \begin{tikzcd}[ampersand replacement=\&,cramped]
            X \&\& {F(X)} \\
            \\
            \&\& Y
            \arrow["{\eta_{X}}", color={solarized-red}, from=1-1, to=1-3]
            \arrow["{f \comp \eta_{X}}"', from=1-1, to=3-3]
            \arrow["f", dotted, from=1-3, to=3-3]
        \end{tikzcd}
    \]
\end{definition}
More concretely,
we are asking for a bijection between the set of homomorphisms from the free algebra to any other algebra,
and the set of functions from the carrier set of the free algebra to the carrier set of the other algebra.
%
In other words, there should be no more data in homomorphisms out of the free algebra than there is in functions out of
the carrier set, which is the property of \emph{freeness}.
%
The inverse operation to post-composition with $\eta_X$ is the \emph{extension} of a function to a homomorphism,
which is common in the functional programming and recursion schemes literature, and known as a \emph{fold},
or a \emph{catamorphism}.
%
\begin{definition}[Universal extension]
    \label{def:universal-extension}
    The universal extension of a function $f : X \to Y$ to a homomorphism out of the free $\sig$-algebra on $X$ is written
    as $\ext{f} : \str{F}(X) \to \str{Y}$.
    %
    It satisfies the identities:
    \begin{itemize}
        \item $\ext{f} \comp \eta_X \htpy f$,
        \item $\ext{\eta_{X}} \htpy \idfunc_{\str{F}(X)}$,
        \item $\ext{(\ext{g} \comp f)} \htpy \ext{g} \comp \ext{f}$.
    \end{itemize}
\end{definition}

The universal property of the free algebra construction makes it unique up to (unique) isomorphism.
%
This can be easily verified with the following lemma.
\begin{lemma}
    \label{lem:free-algebras-unique}
    Suppose that $\str{F}(X)$ and $\str{G}(X)$ are both free $\sig$-algebras on $X$.
    %
    Then, it follows that $\str{F}(X) \eqv \str{G}(X)$, natural in $X$.
\end{lemma}
\begin{proofsketch}
    The maps in each direction are given by extending $\eta_X$ for each free algebra construction:
    $\ext{G\fdot\eta_{X}} : \str{F}(X) \to \str{G}(X)$, and vice versa.
    %
    Finally, using~\cref{def:universal-extension}, we have
    \(
    \ext{F\fdot\eta_{X}} \comp \ext{G\fdot\eta_{X}} \htpy
    \ext{(\ext{F\fdot\eta_{X}} \comp G\fdot\eta_{X})} \htpy
    \ext{F\fdot\eta_{X}} \htpy
    \idfunc_{\str{F}(X)}
    \).
    % \vc{Mention SIP.}
\end{proofsketch}
% \todo{Revisit.}
The free algebra construction automatically turns $F$ into an endofunctor on $\Set$,
where the action on functions is given by:
$X \xto{f} Y \mapsto F(X) \xto{\ext{(\eta_{Y} \comp f)}} F(Y)$.
%
Further, this is gives a monad on $\Set$, where the unit is given by $\eta$,
and the multiplication is given by $\mu_{X}\colon F(F(X)) \xto{\idfunc_{F(X)}} F(X)$.
%
By left adjoints preserve colimits, we have:
\begin{proposition}
    \label{prop:free-algebra-colimits}
    \leavevmode
    \begin{itemize}
        \item $\str{F}(\emptyt) \eqv \unitt$, is a zero object in $\sigAlg$,
        \item $\str{F}(X + Y)$ is the coproduct of $\str{F}(X)$ and $\str{F}(Y)$ in $\sigAlg$:
              \[
                  \sigAlg(\str{F}(X + Y), \str{Z}) \eqv
                  \sigAlg(\str{F}(X), \str{Z}) \times \sigAlg(\str{F}(Y), \str{Z})
                  \enspace.
              \]
    \end{itemize}
\end{proposition}

So far, we've only discussed abstract properties of free algebras, but not actually given a construction!
%
In type theory, \emph{free} constructions are often given by inductive types,
where the constructors are the pieces of data that freely generate the structure,
and the type-theoretic induction principle enforces the category-theoretic universal property.

\begin{definition}[Construction of Free Algebras]
    \label{def:free-algebra-construction}
    The free $\sig$-algebra on a type $X$ is given by the inductive type:
    % \vc{$F$ is also used for free, change $\Sig$ functor to $\Sigma$?}
    \begin{code}
data Tree (X : UU) : UU where
  leaf : X -> Tree X
  node : F$_\sig$(Tree X) -> Tree X
    \end{code}
\end{definition}
The constructors \inline{leaf} and \inline{node} are, abstractly,
the generators for the universal map, and the algebra map, respectively.
%
Concretely, this can be seen as the type of abstract syntax trees for terms in the signature $\sig$,
-- the leaves are the free variables, and the nodes are the branching operations of the tree,
marked by the operations in the signature $\sig$.

\begin{example}
A tree for $\sigMon$ with the carrier set $\Nat$ would look like:
\begin{center}
\scalebox{0.7}{
% https://q.uiver.app/#q=WzAsNixbMiwyLCIrIl0sWzIsMywiNCJdLFszLDEsIisiXSxbMiwwLCIxIl0sWzQsMCwiMSJdLFswLDAsIjIiXSxbMCwxXSxbMiwwXSxbMywyXSxbNCwyXSxbNSwwXV0=
\begin{tikzcd}[ampersand replacement=\&,cramped]
	2 \&\& 1 \&\& 1 \\
	\&\&\& {+} \\
	\&\& {+} \\
	\&\& 4
	\arrow[from=3-3, to=4-3]
	\arrow[from=2-4, to=3-3]
	\arrow[from=1-3, to=2-4]
	\arrow[from=1-5, to=2-4]
	\arrow[from=1-1, to=3-3]
\end{tikzcd}
}
\hspace{2em}
% https://q.uiver.app/#q=WzAsMixbMCwwLCJlIl0sWzAsMiwiMCJdLFswLDFdXQ==
\begin{tikzcd}[ampersand replacement=\&,cramped]
	e \\
	\\
	0
	\arrow[from=1-1, to=3-1]
\end{tikzcd}
\end{center}
    % Some example trees for $\sigMon$, with leaves drawn from $\{\rwb\}$.
    % %
    % \todo{Expand the signature functor and show how this produces binary trees.}
\end{example}

\begin{proposition}
    \label{prop:free-algebra-construction-is}
    \inline{(Tree(X), leaf)} is the free $\sig$-algebra on $X$, for any set $X$.
\end{proposition}

\subsection{Equations}
\label{sec:universal-algebra:equations}

The algebraic framework of universal algebra we have described so far only captures operations, not equations.
%
These algebras are \emph{lawless} (or \emph{wild} or \emph{absolutely free}) --
saying the $\SigMon$-algebras are monoids, or $\str{F}_{\sigMon}$-algebras are free monoids is not justified.
%
For example, these two trees of $(\Nat, +)$ should be identified as equal, but they're not.
\begin{center}
    \scalebox{0.7}{
    % https://q.uiver.app/#q=WzAsNCxbMSwxLCIrIl0sWzAsMCwiMSJdLFsyLDAsIjIiXSxbMSwyLCIwIl0sWzEsMF0sWzIsMF0sWzAsM11d
\begin{tikzcd}[ampersand replacement=\&,cramped]
	1 \&\& 2 \\
	\& {+} \\
	\& 0
	\arrow[from=1-1, to=2-2]
	\arrow[from=1-3, to=2-2]
	\arrow[from=2-2, to=3-2]
\end{tikzcd}
    }
\hspace{2em}
    \scalebox{0.7}{
    % https://q.uiver.app/#q=WzAsNCxbMSwxLCIrIl0sWzAsMCwiMSJdLFsyLDAsIjIiXSxbMSwyLCIwIl0sWzEsMF0sWzIsMF0sWzAsM11d
\begin{tikzcd}[ampersand replacement=\&,cramped]
	2 \&\& 1 \\
	\& {+} \\
	\& 0
	\arrow[from=1-1, to=2-2]
	\arrow[from=1-3, to=2-2]
	\arrow[from=2-2, to=3-2]
\end{tikzcd}
    }
\end{center}
%
To impose equations on the operations, we adopt the point of view of equational logic.

\begin{definition}[Equational Signature]
    An equational signature, denoted $\eqsig$, is a (dependent) pair consisting of:
    \begin{itemize}
        \item a set of names for equations, $\eqop : \Set$,
        \item an arity of free variables for each equation, $\eqfv : \eqop \to \Set$.
    \end{itemize}
\end{definition}

\begin{example}
    The equational signature for monoids $\eqsigMon$ is encoded as:
    $(\Fin[3],\lambda \{0 \mapsto \Fin[1] ; 1 \mapsto \Fin[1] ; 2 \mapsto \Fin[3] \})$.
    %
    There are three equations for monoids -- the left and right unit laws, and the associativity law,
    which is a 3-element set $\{ \term{unitl}, \term{unitr}, \term{assoc} \}$.
    %
    The two unit laws require one free variable, or parameter (that is universally quantified):
    $\term{unitl} : \forall x,\, e \mult x \id x$, and $\term{unitr} : \forall x,\, x \mult e \id x$.
    %
    The associativity law requires three free variables:
    $\term{assoc} : \forall x, y, z,\, (x \mult y) \mult z \id x \mult (y \mult z)$.
    %
    This is what is being encoded by the ``arity of free variables'' function.
\end{example}

Just like the signature functor~\cref{def:signature-functor}, this produces an equational signature functor on $\Set$.
\begin{definition}
    \label{def:equational-signature-functor}
    \begin{align*}
        \EqSig \colon \Set & \to \Set                             \\
        X                  & \mapsto \dsum{e:\eqop}{X^{\eqfv(e)}} \\
        X \xto{f} Y        & \mapsto
        \dsum{e:\eqop}{X^{\ar(e)}}
        \xto{(e, \blank \comp f)}
        \dsum{e:\op}{Y^{\ar(e)}}
    \end{align*}
\end{definition}

To build equations out of this signature,
we need to use the $\sig$-operations and construct an honest equation -- a system of equations.
%
Informally, this is a pair of trees for each equation,
one for the left-hand side and one for the right-hand side of the equation,
that can use the free variables available to the equation.
%
Using the abstraction notion of the equational signature functor and the free algebra functor,
this is simply a natural transformation!

\begin{definition}[System of Equations]
    A system of equations over a signature $(\sig,\eqsig)$, is a pair of natural transformations:
    \[
        \eqleft, \eqright : \EqSig \natto \str{F}_{\sig} \enspace.
    \]
    Concretely, for any set $V$,
    this gives a pair of trees $\eqleft_{V}, \eqright_{V} : \EqSig(V) \to \str{F}_{\sig}(V)$,
    and naturality ensures correctness of renaming.
\end{definition}

\begin{example}
The equational signature $\sigma_{\mathsf{Mon}}$ has
the set of equation symbols $eq = \{\term{unitl}, \term{unitr}, \term{assocr}\}$
(or $\Fin[3]$ or $\mathbf{3}$),
and the arity function
$fv : eq \rightarrow \mathcal{U} = \{\term{unitl} \mapsto \mathbf{1}, \term{unitr} \mapsto \mathbf{1}, \term{assocr} \mapsto \mathbf{3}\}$.

To show $(\Nat,0,+) \entails \term{Mon}$:
\begin{align*}
\term{unitl}  & : \forall (\rho : \Nat^{\Fin[1]}). \, \rho(0) + 0 \id \rho(0) \\
\term{unitr}  & : \forall (\rho : \Nat^{\Fin[1]}). \, 0 + \rho(0) \id \rho(0) \\
\term{assocr} & : \forall (\rho : \Nat^{\Fin[3]}). \, (\rho(0) + \rho(1)) + \rho(2) \id \rho(0) + (\rho(1) + \rho(2))
\end{align*} 
\end{example}
Finally, given a $\sig$-structure $\str{X}$,
we can say that it \emph{satisfies} the system of equations $T_{(\sig,\eqsig)}$.
%
Informally, we need to assign a value to each free variable in the equation, picking them out of the carrier set,
which is the function $\rho : V \to X$.
%
Given such an assignment, we can evaluate the left and right trees of the equation,
simply by extending $\rho\,$ (using~\cref{def:universal-extension}),
that is by construction, a homomorphism from the free algebra $\str{F}(V)$ to the structure $\str{X}$.
%
To satisfy an equation, these two evaluations should agree!
%

\begin{definition}[$\str{X} \entails T$]
    A $\sig$-structure $\str{X}$ satisfies the system of equations $T_{(\sig,\eqsig)}$ if for every set $V$,
    and every assignment $\rho : V \to X$, the following diagram commutes:
    % https://q.uiver.app/#q=WzAsMyxbMiwwLCJcXHN0cntGfShWKSJdLFs0LDAsIlxcc3Rye1h9Il0sWzAsMCwiXFxFcVNpZyhWKSJdLFswLDEsIlxcZXh0e1xccmhvfSJdLFsyLDAsIlxcZXFsZWZ0X3tWfSIsMCx7Im9mZnNldCI6LTN9XSxbMiwwLCJcXGVxcmlnaHRfe1Z9IiwyLHsib2Zmc2V0IjozfV1d
    \[\begin{tikzcd}[ampersand replacement=\&,cramped]
            {\EqSig(V)} \&\& {\str{F}(V)} \&\& {\str{X}}
            \arrow["{\ext{\rho}}", from=1-3, to=1-5]
            \arrow["{\eqleft_{V}}", shift left=3, from=1-1, to=1-3]
            \arrow["{\eqright_{V}}"', shift right=3, from=1-1, to=1-3]
        \end{tikzcd}\]
\end{definition}

% \todo{Full subcategory of $\sigAlg$ is a variety of $(\sig,\eqsig)$-algebras.}
% \todo{Construction of the free object, requires non-constructive principles~\cite{blassWordsFreeAlgebras1983}.}
% \todo{HITs are another way to write higher generators for equations~\cite{univalentfoundationsprogramHomotopyTypeTheory2013}.}

We have now developed the basic notions of universal algebra and equational logic,
and we only consider the constructions of free monoids and free commutative monoids in the next sections.
