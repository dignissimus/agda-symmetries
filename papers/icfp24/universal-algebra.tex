\section{Universal Algebra}
\label{sec:universal-algebra}

To explain our results, we first develop some basic notions from universal algebra and equational logic~\cite{birkhoffStructureAbstractAlgebras1935} -- this gives us a vocabulary and framework.
.~\vc{How much is necessary and how much is overkill?}
%
This is standard following any textbook on model theory of logic, and predates the Lawvere theory or abstract clone point of view.

Universal algebra is the abstract study of algebraic structures, which have (algebraic) operations and (universal) equations.
%
We keep a running example of monoids in mind, while explaining and defining the abstract concepts.
%
For example, a monoid, is a carrier set, with an identity element (or a nullary operation), and a binary multiplication operation. In the language of universal algebra, such a structure is formalized by giving a signature of operations, and a structure being a carrier set with functions that interpret these operations. We describe such a framework for universal algebra in HoTT, as follows.

\begin{definition}[Signature]
    A signature, denoted $\sigma$, is a pair consisting of:
    \begin{itemize}
        \item a set of operations, $\term{op}\colon \Set$,
        \item an arity function, $\term{ar}\colon \term{op} \to \Set$.
    \end{itemize}
\end{definition}

\begin{example}
    The signature for monoids $\sigma_{\Mon}$ is given by: 
    $(\Fin[2],\lambda \{0 \mapsto \emptyt ; 1 \mapsto \boolt \})$.
    Informally, the set of operations is the two-element set: $\{e,\mult\}$ which is written as $\Fin[2]$, and the arity functions picks out a finite type denoting the (finite) arity of each operation.
\end{example}

Every signature $\sigma$ induces a signature functor on $\Set$, written $F_{\sigma}$.
\begin{definition}[Signature functor]
    \[\begin{aligned}
            F_{\sigma} \colon \Set &\to \Set \\
            X &\mapsto \dsum{o:\term{op}}{X^{\term{ar}(o)}} \\
            X \xto{f} Y &\mapsto \dsum{o:\term{op}}{X^{\term{ar}(o)} \xto{{f}^{\term{ar}(o)}} Y^{\term{ar}(o)}}
      \end{aligned}\]
    \todo{fix dsum macro}
\end{definition}

A $\sigma$-structure, say $\str{X}$, is given by a carrier set, with functions for each operation symbol. Formally, this is exactly an algebra for the $F_{\sigma}$ functor.
\begin{definition}[Structure]
    A $\sigma$-structure $\str{X}$ is an $F_{\sigma}$ algebra, that is:
    \begin{itemize}
        \item a carrier set $X$, and
        \item an algebra map $\alpha_{X}\colon F_{\sigma}(X) \to X$.
    \end{itemize}
\end{definition}

\begin{example}
    A well-known example of a monoid is the natural numbers $\Nat$, with $0$ and $+$.
    To see how this is an $F_{\sigma_{\Mon}}$-algebra, notice that
    \[
        \begin{array}{rl}
            \alpha_{\Nat} &= \dsum{o:\Fin[2]}{\Nat^{\term{ar}(o)}} \to \Nat \\
                          &\eqv (\Nat^{\Fin[0] + \Fin[2]}) \to \Nat \\
                          &\eqv (\Nat^{\Fin[0]} \to \Nat) \times (\Nat^{\Fin[2]} \to \Nat) \\
                          &\eqv (\unitt \to \Nat) \times (\Nat \times \Nat \to \Nat)
        \end{array}
    \]
    which is given by the $0$ and $+$ operations, respectively.
\end{example}

\begin{definition}[Homomorphism]
    A homomorphism is
\end{definition}

For any object \( \mathfrak{Y} \) in $\sigma$-Alg, $(\blank) \circ \eta_X$ is an equivalence:

\begin{figure}[H]
    \centering
    % https://q.uiver.app/#q=WzAsMyxbMCwwLCJYIl0sWzIsMCwiVShBKSJdLFsyLDIsIlUoQikiXSxbMCwxLCJpIiwwLHsiY29sb3VyIjpbMSwxMDAsNjBdfSxbMSwxMDAsNjAsMV1dLFswLDIsImciLDJdLFsxLDIsIlUoZikiLDAseyJzdHlsZSI6eyJib2R5Ijp7Im5hbWUiOiJkb3R0ZWQifX19XV0=
    \[\begin{tikzcd}[ampersand replacement=\&]
    	\mathfrak{F}(X) \\
    	\\
    	\mathfrak{Y}
    	\arrow["f", dotted, from=1-1, to=3-1]
    \end{tikzcd}
    \mapsto
    \begin{tikzcd}[ampersand replacement=\&]
    	X \&\& {F(X)} \\
    	\\
    	\&\& {Y}
    	\arrow["\eta_X", color={rgb,255:red,255;green,54;blue,51}, from=1-1, to=1-3]
    	\arrow["f \circ \eta_X"', from=1-1, to=3-3]
    	\arrow["{f}", dotted, from=1-3, to=3-3]
    \end{tikzcd}\]
    \caption{Universal property of free algebras}
    \label{fig:universal-property}
\end{figure}



In this paper, we are interested in the constructions of free monoids and free commutative monoids.