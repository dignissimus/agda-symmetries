\section{Sorting}
\label{sec:sorting}

\vc{This theorem works for any presentation.}
\vc{Different presentations can be advantageous to different kinds of sorting algorithms.}

Consider the canonical map $q : \LL(A) \to \MM(A)$ given by $\ext{\eta_A}$. $q$ would be a surjective
function which "forgets" the order on a free monoid and turns it into a free commutative monoid.
The axiom of choice says every surjection $f$ has a section $s$ such that $\forall x.\, f(s(x)) = x$.

\begin{figure}[H]
    \centering
    \scalebox{1.3}{
        % https://q.uiver.app/#q=WzAsMixbMCwwLCJcXExMKEEpIl0sWzMsMCwiXFxNTShBKSJdLFsxLDAsInMiLDAseyJjdXJ2ZSI6LTF9XSxbMCwxLCJxIiwwLHsiY3VydmUiOi0xfV1d
    \begin{tikzcd}[ampersand replacement=\&,cramped]
    	{\LL(A)} \&\&\& {\MM(A)}
    	\arrow["s", curve={height=-6pt}, from=1-4, to=1-1]
    	\arrow["q", curve={height=-6pt}, two heads, from=1-1, to=1-4]
    \end{tikzcd}
    }
    \caption{Relationship of $\LL(A)$ and $\MM(A)$}
    \label{fig:enter-label}
\end{figure}

We can now study sort functions as sections and their relationship with total orders.
First, we recall the axioms of total order are:
\begin{itemize}
    \item reflexivity: $x \leq x$
    \item transitivity: if $x \leq y$ and $y \leq z$, then $x \leq z$
    \item antisymmetry: if $x \leq y$ and $y \leq x$, then $x = y$
    \item totality: forall $x$ and $y$, we have merely either $x \leq y$ or $y \leq x$ 
\end{itemize}

In the context of this paper, we also want to make a distinction between "decidable total order"
and just "total order". A decidable total order should also satisfy
\begin{itemize}
    \item decidability: forall $x$ and $y$, we have either $x \leq y$ or $\neg(x \leq y)$ 
\end{itemize}

This is a stronger version of the totality axiom, where with totality we have 
either $x \leq y$ or $y \leq x$ merely as a proposition, but decidability allows us to actually
compute if $x \leq y$ is true.

\begin{proposition}
Assume there is a decidable total order on $A$. There is a sort function $s: \MM(A) \to \LL(A)$
which constructs a section to $q : \LL(A) \twoheadrightarrow \MM(A)$
\end{proposition}

We can easily construct such a section by any sorting algorithm. In our formalization we chose
insertion sort due to its inductive properties and simple definition. It is easy to see how this
proposition holds: $q(s(xs))$ orders an unordered list $xs$ by $s$, and re-discard the order again by
$q$, the act of imposing then forgetting an order on $xs$ does nothing, therefore $q(s(xs)) = xs$ holds.

\subsection{Order from section}
At this point one may ask if we can construct a section from order, can we construct an order from section?
Indeed, just by the virtue of $s$ being a section, we can almost construct a relation that satisfies
all axioms of total order except transitivity. 

\begin{definition}
Given a section $s$, we define $\term{least}(xs) := \term{head}(s(xs))$.
Further, we define $x \preccurlyeq y := \term{least}(\{x, y\}) = x$. 
\end{definition}

Conceptually, $\term{least}$ sorts $xs$ by $s$, and pick the first element of the sorted list.
With this we can easily prove the following properties of $\preccurlyeq$.

\begin{theorem}
$\preccurlyeq$ is reflexive.
\end{theorem}
For all $x$, $\term{least}(\{x, x\})$ must be $x$, therefore $x \preccurlyeq x$.

\begin{theorem}
$\preccurlyeq$ is antisymmetric.
\end{theorem}
For all $x$ and $y$, given $x \preccurlyeq y$ and $y \preccurlyeq x$,
we have $\term{least}(\{x, y\}) = x$ and $\term{least}(\{y, x\}) = y$.
Since $\{x, y\} = \{y, x\}$, $\term{least}(\{x, y\}) = \term{least}(\{y, x\})$,
therefore we have $x = y$.

\begin{theorem}
$\preccurlyeq$ is total.
\end{theorem}
For all $x$ and $y$, $\term{least}(\{x, y\})$ is merely either $x$ or $y$,
therefore we have merely either $x \preccurlyeq y$ or $y \preccurlyeq x$.


\vc{Case study: we test these axioms on a few sorting algorithms.}