\section{Type Theory}\label{sec:type-theory}
As we have explained, our work is formalized in Cubical Agda and Cubical Type Theory,
which is a variant of Homotopy Type Theory that is designed to preserve
computational properties of type theory.
We refer the readers to other works such as~\cite{vezzosiCubicalAgdaDependently2019}
and~\cite{cohenCubicalTypeTheory2018} for a more in-depth explanation on Cubical Type Theory
and how we can program in Cubical Agda.
We also give an overview of relevant features
in Cubical Type Theory which normal type theory (namely MLTT) lacks, and how these features
are used within the scope of our work.
\subsection{Dependent Types}
We first review some basic ideas from type theory. We assume the readers
have experience with dependent typed programming, so this would only
serves as a refresher.
\subsubsection{$\Pi$-types}
$\Pi$-types generalizes function types.
These types capture the idea of functions that can take different types
as input or return different types as output depending on the input.
The type of a function that accepts an argument $x: A$ and returns
an element of type $B(x)$, where $B$ might depend on $x$,
can be expressed as $(x: A) \to B(x)$ or $\Pi_{(x: A)} B(x)$.

\subsubsection{$\Sigma$-types}
$\Sigma$-types generalizes product types. 
These types represent tuples where the type of the second component
depends on the first. An element of the type
$\Sigma_{(x: A)} B(x)$ is a pair $(a, b)$
where $a$ is of type $A$ and $b$ is of type $B(a)$.

\subsubsection{Identity types}
The identity type of a type $A$ expresses equality in type theory.
If $a$ and $b$ are elements of type $A$, then the type expressing that
$a$ is equal to $b$ is often written $a =_A b$. Optionally, the type can
be omitted for brevity, therefore written just as $a = b$. This is
often useful to state when two elements should be equal but they are not
definitionally equal. For example, given $n : \Nat$, the elements
$n + 0 : \Nat$ and $0 + n : \Nat$ should be the same, however depending
on the implementation of $+$ they may not be definitionally equal,
therefore $B(n + 0)$ and $B(0 + n)$ would give us different types
definitionally. Having a type $n + 0 =_\Nat 0 + n$ would allow us to
convert $x : B(n + 0)$ to $x : B(0 + n)$ and vice versa, therefore
solving the over-restriction of definitional equality.

\subsubsection{Propositions as Types}
The Curry-Howard correspondence gives us the propositions-as-types
principle, which states that propositional logic can be interpreted
as types:
\begin{itemize}
    \item A type $P$ can be viewed as a proposition, and a term of that
    type is a proof of the proposition.
    \item $\Pi$-types correspond to universal quantification or implications.
    A function $f : P \to Q$ can be thought of as "$P$ implies $Q$", and
    a function $f : (a : A) \to P(a)$ can be thought of as "for all $a$
    of type $A$, $P(a)$ holds".
    \item $\Sigma$-types correspond to existential quantification.
    An element of $\Sigma_(a : A)\,P(a)$ can be thought of as a proof
    that states "there exists an $a$ of type $A$ such that $P(a)$ holds".
\end{itemize}

\subsection{Homotopy Types}
In Homotopy Type Theory, types are assigned a homotopy level
(often abbreviated h-level). I will explain the concept of h-level
since it is important to understand the terminlogy used in the rest
of the dissertation.
\subsubsection{Contractible Type}
A type $A$ is said to be contractible iff there is an element
$a : A$, refered to as the \emph{center of contraction}, such that
for all $x: A$, we have $a = x$. More formally:
\begin{align*}
    \term{isContr}(A) := \Sigma_{(a : A)}\Pi_{(x : A)}(a =_A x).
\end{align*}
Contractible types are assigned the h-level 0.
One example of such a type would be the unit type $\unitt$.
In fact, all contractible types are equivalent to $\unitt$.
\subsubsection{Mere proposition}
Going one dimension higher, we have mere propositions. A type $A$
is said to be a mere proposition iff all elements of $A$ are equal.
We give two formal and equivalent definitions:
\begin{align*}
    \term{isProp}(A) & := \Pi_{(x,y : A)}(x =_A y).\\
    \term{isProp}(A) & := \Pi_{(x,y : A)}\term{isContr}(x =_A y).\\
\end{align*}
Mere propositions are assigned the h-level 1.
We use $\hProp := \Sigma_\UU \term{isProp}.$ to denote the type for
all types that are propositions.
Examples of such types
are the unit type $\unitt$ and the empty type $\emptyt$. In HoTT,
we use the adverb \emph{merely} to describe propositions represented
as a mere propositional type. We also introduce the idea of
propositional truncation, where a type $A$ is truncated to 
a mere propositional type $\norm{A}$. This can be done with a higher
inductive type (elaborated more in~\cref{sec:HIT}):

\begin{code}
data [[_]] (A : UU) : UU where
    [_]  : A -> [[A]]
    trunc : (x, y : [[A]]) -> x = y
\end{code}
One example usage of propositional truncation is the definition of
\emph{mere existence}, where $\exists (x: A).\,P(x)$ is represented
as $\norm{\Sigma_{(x:A)}P(x)}$. This differs from $\Sigma_{(x:A)}P(x)$
in that mere existence only contains information whether such an $x$
that satifies $P$ exists, whereas the $\Sigma$ type contains more information,
namely the value of the actual element $x$. This makes it such that
with a mere existential proof of $x$ we would only be able to use $x$
to prove other mere propositions, but we would not be able to define
any functions that actually depend on the value of $x$. For example,
consider a function $f : (x : A) \to P(x) \to \Nat$ which takes an $x: A$
that satisfies $P$ and return a $\Nat$. Given $\exists x.\,P(x)$ we would
not be able to apply $f$ on $x$, since $\Nat$ is not a proposition. 
On the other hand, assuming we have another mere proposition
$Q : A \to \hProp$, and a proof that $\forall y.\,P(y) \to Q(y)$, 
we can use $\exists x.\,P(x)$ to prove $\exists x.\,Q(x)$.

\subsubsection{Sets}
Going one dimension higher, we have sets. A type $A$ is said to be a set
iff the identity type of $A$ is a mere proposition.
\begin{align*}
    \term{isSet}(A) := \Pi_{(x,y : A)}\term{isProp}(x =_A y).
\end{align*}
Sets are assigned the h-level 2.
We use $\hSet := \Sigma_\UU \term{isSet}.$ to denote the type for
all types that are sets.
Examples of sets would be $\Nat$, the boolean type $\boolt$, and also
the unit and empty type $\unitt$ and $\emptyt$. The type for all types
that are mere propositions $\hProp$ is also a set.

We note that in type theories with K-axiom or uniqueness of identity proofs
(UIP), all types are sets. For example, in Idris2~\cite{brady_idris_2021},
we can prove this
by pattern matching on proofs:
\begin{code}
uip : (A : Type) -> (x, y : A) -> (p, q : x = y) -> p = q
uip A x x Refl Refl = Refl
\end{code}

The K elimination rule allows us to pattern match on identity types and force
them to be $\term{Refl}$, thus allowing us to prove $p = q$ since both
$p$ and $q$ are $\term{Refl}$. HoTT, however, does not assume the K-axiom,
therefore this would be invalid in HoTT.

In the dissertation we would be primarily working with sets
(and mere propositions since all mere propositions are sets).
There are higher dimensional types, but generalizing the work to
higher dimensions is currently left for future works.

\subsubsection{Groupoid}
Going one dimension higher, we have groupoids. A type $A$ is said to be a
groupoid iff the identity type of $A$ is a set.
\begin{align*}
    \term{isSet}(A) := \Pi_{(x,y : A)}\term{isSet}(x =_A y).
\end{align*}
Groupoids are assigned the h-level 3. All sets are groupoids,
and the type for all types that are sets $\hSet$ is also a groupoid.
For example, consider the boolean type $\boolt : \hSet$. There are two
different elements of $\boolt =_\hSet \boolt$, which generated by
univalence (elaborated more in~\cref{types:univalence}) from two
different equivalence we can define on $\boolt$ to $\boolt$, the
identity function $id$ and the negate functon $not$.
This shows that $\boolt =_\hSet \boolt$ is \emph{not} a mere proposition,
therefore $\hSet$ is \emph{not} a set. This shows how K-axiom is
incompatible with univalence and HoTT.

\subsubsection{n-Types}
We can define further higher dimensional types by induction, giving us
2-groupoids, 3-groupoids, and so on. More generally, we can define
n-types as below, with $n$ starting at -2.
\begin{align*}
    \term{is-n-type}(A) := \begin{cases}
        \term{isContr}(A) & \text{if $n = -2$} \\
        \prod_{x, y : A}\term{is-(n-1)-type}(x =_A y) & \text{otherwise}
    \end{cases}
\end{align*}

Somewhat confusingly, (-2)-types (contractible types) have h-level 0,
and (-1)-types (mere propositions) have h-level 1, and so on.
In the lingo of HoTT, when we say n-types we start counting from -2,
but for h-levels we start counting from 0. This is because when
the concept of h-level is invented Voevodsky chose to start counting
from 0~\cite{voevodsky_univalent_2010},
but the idea of n-trunction originates earlier from $\inf$-category theory,
where truncation level starts from -2~\cite{lurie_higher_2008}.

\subsection{Function Extensionality}
Function extensionality $\term{funExt}$ states that given two functions $f$ and $g$,
$\term{funExt}: \forall x.\,f(x) = g(x) \to f = g$. MLTT by itself does not have $\term{funExt}$,
instead it has to be postulated as an axiom, thereby losing canonicity, in other words,
we would not be able to compute any constructions of elements that involve $\term{funExt}$
in its construction. In Cubical Type Theory, we can derive $\term{funExt}$
as a theorem while also preserving canoncity, therefore we can compute with constructions
involving $\term{funExt}$!

Within the scope of our work, $\term{funExt}$ is heavily used in
~\cref{mon:array} and~\cref{cmon:bag}, where a $n$-element array $A^n$ is defined as lookup functions
$\Fin[n] \to A$. Therefore, to prove two arrays are equal, we need to show that two functions would be
equal, which is impossible to do without $\term{funExt}$. We also would not be able to normalize
any constructions of arrays which involve $\term{funExt}$ if it is postulated as an axiom, therefore
the computational property of Cubical Type Theory is really useful for us.

\subsection{Higher Inductive Types}\label{sec:HIT}
Higher inductive types
allow us to extend inductive types to not only allow point constructors
but also path constructors, essentially equalities between elements of the HIT. One such example
would be the following definition of $\mathbb{Z}$:

\vspace{-1em}
\begin{code}
data $\mathbb{Z}$ : UU where
    pos : (n : $\mathbb{N}$) -> $\mathbb{Z}$
    neg: (n : $\mathbb{N}$) -> $\mathbb{Z}$
    posneg : pos 0 == neg 0
\end{code}
\vspace{1em}

The integers are often represented with a $\term{pos} : \Nat \to \mathbb{Z}$
and $\term{negsuc} : \Nat \to \mathbb{Z}$, where natural numbers
are mapped into the integers via the $\term{pos}$ constructor, and negative numbers are constructed
by mapping $n : \Nat$ to $-(n + 1)$. The shifting by one is done to avoid having duplicate elements
for 0, which can easily lead to confusion. HIT allows us to define integers more naturally by saying
$\term{pos}(0) = \term{neg}(0)$, avoiding the confusing shift-by-one hack.

With HITs we can define a general data type for set quotients:
\begin{code}
data _/_ (A : UU) (R : A -> A -> UU) : UU where
    [_]  : A -> A / R
    q/ : (a b : A) -> (r : R a b) -> [ a ] == [ b ]
    trunc : (x y : A / R) -> (p q : x = y) -> p = q
\end{code}

The $\term{q/}$ constructor says if $a$ and $b$ can be identified by a relation $R$, then
they should belong to the same quotient generated by $R$.
One might also notice the $\term{trunc}$ constructor, which says and proofs of $x =_{A/R} y$ are equal.
This constructor basically truncates the $\term{\_/\_}$ type down to set, so that we do not have to
concern ourselves with higher paths generated by the $\term{q/}$ constructor.
We give a more precise definition of "set" in Homotopy Type Theory and examples of types that are
higher dimension than sets in~\cref{types:univalence}.

In our work, higher inductive types and set quotients are used extensively to define commutative
data structures, which we would demonstrate in~\cref{sec:commutative-monoids}. In MLTT we can only
reason with quotients and commutativity by setoids, which comes with a lot of proof burden
since we cannot work with the type theory's definition of equalities and functions directly,
instead we have to define our own equality relation and define our own type for setoid homomorphisms,
giving rise to the infamous setoid hell.

\subsection{Univalence}\label{types:univalence}
In MLTT we cannot construct equalities between types. Univalence, the core of Homotopy Type Theory,
gives us equalities between types by the univalence axiom. To see how this is useful,
consider $A, B : \mathcal{U}, P : \mathcal{U} \rightarrow \mathcal{U}, A = B$, we can
get $P(B)$ from $P(A)$ by transport (or substitution).  

To define univalence, we first need to define the notion of equivalence:
\begin{definition}[Equivalence]
    A function $f : A \to B$ is said to be an equivalence
    function iff it has a left inverse $g : B \to A$ and right inverse 
    $h : B \to A$ such that
    $(\forall x.\, g(f(x)) = x) \land (\forall x.\, f(h(x)) = x)$. 
    Two types $A$ and $B$ are said to be equivalent iff there exists
    an equivalence function $f : A \to B$. 
\end{definition}

We note that the definition of equivalence is very similar to
an isomorphism, only differing in that an equivalence can have
different functions for left and right inverse, whereas an isomorphism
would have the same function, therefore making it just an (ordinary) inverse.

Assuming $A$ (and subsequently $B$ by equivalence) is a $\hSet$,
$f$ being a equivalence function would imply it is an isomorphism and 
vice versa, thus making both definitions the same. However, for higher
dimensional types, these two definitions would not necessarily imply
each other.
More explicitly: let $\term{qinv}(f)$
be the witness on $f$
stating $f$ is an isomorphism, and $\term{isequiv}(f)$ be the witness on
$f$ stating $f$ is an equivalence function.
\begin{align*}
\term{qinv}(f) & = \Sigma_{g : B \to A}(f \comp g \sim id)\times(g \comp f \sim id) \\
\term{isequiv}(f) & = (\Sigma_{g : B \to A}(f \comp g \sim id))\times(\Sigma_{h : B \to A}(h \comp f \sim id))
\end{align*}
$\term{isequiv}(f)$ is a mere proposition while $\term{qinv}(f)$ is not
necessarily a mere proposition.
We refer the readers
to~\cite[Chapter 4.1]{univalentfoundationsprogramHomotopyTypeTheory2013} for a proof.

\begin{definition}[Univalence axiom]
    $(A = B) \simeq (A \simeq B)$
\end{definition}
Essentially, if we can construct an equivalence between two types $A$ and $B$,
we can obtain a equality of the types $A$ and $B$. While in HoTT we cannot
compute with elements constructed with univalence since it is postulated as an
axiom, Cubical Type Theory allows us to derive univalence as a computable
theorem, therefore we can transport functions and proofs from one type to
another freely without worrying about terms not being able to normalize!

Within the scope of our work, since we have multiple constructions of free monoids
and free commutative monoids, given in~\cref{sec:monoids} and~\cref{sec:commutative-monoids},
having univalence allows us to easily transport proofs and functions from one construction to another.
Another instance where univalence is used is the definition of membership proofs in~\cref{comb:member},
where we want to show to propositions are commutative: i.e. $\forall p, q: \hProp, p \vee q = q \vee p$.
Since $p$ and $q$ are types, we need univalence to show $p \vee q = q \vee p$ are in fact equal.

\section{Notation}\label{sec:notation}
We give a quick overview of the notations used in this paper.

We denote the type of types with $\mathcal{U}$. 
In practice, $\mathcal{U}$ would be indexed by a type level 
à la Russell for consistency, however we opted to omit the type level
for simplicity and clarity.
We use $\Fin[n]$ to denote finite sets of cardinality $n$ in HoTT~\cite{yorgeyCombinatorialSpeciesLabelled2014a}.
This is defined as follows:

\begin{code}
Fin : $\mathbb{N}$ -> UU
Fin n = $\Sigma$[ m $\in$ $\mathbb{N}$ ] (m < n)
\end{code}

There are other ways to define $\Fin$, most notably as an indexed inductive type:
\begin{code}
data Fin : $\mathbb{N}$ -> UU where
    fzero : {k : $\mathbb{N}$} -> Fin (suc k)
    fsuc : {k : $\mathbb{N}$} -> Fin k -> Fin (suc k)
\end{code}


In our construction we opted to use the $\Sigma$-type definition
because cubical Agda does not behave well
when pattern matching on indexed inductive types.

We also use $\times$ to denote product types and $+$ to denote coproduct types.
For mere propositions, we use $\land$ to denote logical and, and $\vee$ to denote logical or.
In Cubical Agda these are defined as propositionally-truncated products and coproducts.
We also use $\exists$ to denote mere existence, which is defined as
propositionally-truncated $\Sigma$-types are shown above.