\label{sec:discussion}

We conclude by discussing some background information regarding the project,
related works, and future directions we can take.

The project was originally meant to be a continuation to the
paper~\cite{choudhuryFreeCommutativeMonoids2023}, where
free commutative monoids are studied under HoTT. We meant to
generalize the study of free commutative monoids on sets to
groupoids, therefore giving us free symmetric monoidal categories.

We decided to rewrite the formalization done in the paper so that
it would have a cleaner codebase, ideally one that would unify
all the proofs related to the universal property of free algebras,
therefore the project evolved into developing a framework for
universal algebra, with the goal to first start from algebra on sets
and eventually generalizing it to groupoids so we can study
symmetric monoidal categories. To demonstrate the effectiveness
of the framework, we surveyed different constructions of free
monoids and free commutative monoids and formalized them under the
framework. Many constructions of free commutative monoids come
from~\cite{joramConstructiveFinalSemantics2023}, which proves
their universal property by isomorphism to a known construction
of free commutative monoid as a HIT. We opted to take a different route,
which is to prove their universal property directly, giving us the new
proof of $\Bag$'s universal property in~\cref{cmon:bag}, so as to
compare the difficulty of proof by isomorphism and direct proofs.
No definite conclusions were drawn, other than that the main difficulty
comes from technical reasons, namely Agda not being able to reduce
a huge $\Bag$ or $\Array$ for reasons we have highlighted
in~\cref{bag:rep}.

After having formalized different constructions of free commutative monoids,
we decided to move on to generalizing the framework to groupoids, and
also prove $\SList$ generalized to groupoids would be a free symmetric
monoidal category. This proved to be quite difficult, due to the lack
of existing literature on this topic and the amount of mathematical background
knowledge involved. Out of concern that it cannot be completed within
half a semester, we moved on to other goals, which was to study the
definition of sorting under the framework, and to investigate its relationship
to the well-ordering theorem, and equivalently, the axiom of choice.

The correctness of sorting algorithms has been studied extensively,
with the most simple specification being that of~\cite{appelVerifiedFunctionalAlgorithms2023},
which is defined for $\List(\Nat) \to \List(\Nat)$, however it can be generalized
to any set $A$ with a total order $\leq$ easily. A more refined study has
been done by~\cite{hinzeSortingBialgebrasDistributive2012}, and further
refined in~\cite{alexandruIntrinsicallyCorrectSorting2023}, in which
sorting algorithms are studied under a categorical framework. Their study
not only studies the extensional correctness of sorting algorithms but
also the algorithm itself, which allows them to derive sorting algorithm
"for free". Our work compliment theirs in that we are not interested in the
computation properties of sorting, but rather the abstract properties of
sorting, independent of a given ordering, with the goal to eventually study
how assuming the existence of sorting functions on sets would imply
constructive taboos, namely the axiom of choice. While we could not draw
the relationship of sorting and axiom of choice due to the lack of time,
we still managed to find new interesting properties of sorting functions
in~\cref{sec:sorting}, where we define order in terms of sorting, as opposed
to defining sorting from orders.

\section{Conclusion}
With Cubical Agda, we have developed a framework for universal algebra
which allows us to prove general properties of free algebras.
We surveyed existing constructions of free monoids and
free commutative monoids and formalized them under the framework,
most notably giving a new proof of $\Bag$'s universal property.
We also study how commutativity impose the notion of unorderedness,
and how free commutative monoids can be used to formalize the notion
of unordered lists. We then study the correctness of sorting functions
from the lens of universal algebra, studying them as the section to
the canonical map from free monoids to free commutative monoids.
We arrive at a new axiomatization of sorting functions, in which
sorting functions are not defined in terms of total order, but rather
defined purely in terms of property of functions.

In the future, it would be interesting to build upon the work done on this
dissertation to explore how our abstract properties of sorting can be
generalized to other inductive types, such as binary trees, which might
give new insight into the constructions of binary search trees. It would
also be interesting to pursue our original goals, such as investigating
the relationship between sorting and axiom of choice, and generalizing the
framework to groupoids so we can study free symmetric monoid categories
in HoTT.