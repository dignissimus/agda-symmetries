\label{sec:universal-algebra}

As explained above, we can formalize the notion of an unordered list
as the free commutative monoid and the idea of a list as the free
monoid. We can then study sorting as a relationship between these
two structures under the framework of universal algebra~\cite{birkhoffStructureAbstractAlgebras1935}.
However, to begin, we would need to explain what an algebra is,
what exactly is the definition of "free".

\section{Algebra}\label{algebra:signature}
First we define very generally what an algebra is. Every algebra
has a signature $\sigma$, which is a (dependent) tuple of:
\begin{itemize}
    \item a set of function symbols $op : \Set$,
    \item and an arity function $ar : op \to \Set$.
\end{itemize}

Here, $op$ gives us a set of names to refer to the operations we can do
in an algebra, and the arity function $ar$ specifies the arity of the operation,
using the cardinality of the output set as the arity. In other definitions,
the arity function may be defined as $ar : op \to \Nat$, which specifies
the arity of an operation $op$ by the output natural number. However,
for generality, we decide to use $\hSet$ and cardinality for arity instead,
so that we can have operations of infinite arity. 

\begin{example}
We can define the signature of monoid as follow: the algebra $\Mon$
has operation symbols $\{ e, \mult \}$, referring to the identity operation
and the multiplication operation. The arity function is defined as:
$ar := \{ e \mapsto \Fin[0], \mult \mapsto \Fin[2] \}$.
\end{example}

We use $X^Y$ to denote a vector of $X$ from now on, with the dimension of
the vector given by the cardinality of $Y$. We now can define the type
$F_\sigma(X) := \Sigma_{(f: op)}X^{ar(f)}$, which would be an endofunctor
on $\Set$. We refer to this endofunctor as the signature endofunctor.
In plain English, $F_\sigma(X)$ is a tuple of an operation symbol $op$
and a vector of $X$, where the dimension of the vector is the arity of $op$.

\begin{definition}\label{algebra:struct}
    A $\sigma$-structure $\mathfrak{X}$ is an $F_\sigma$-algebra, given by a tuple
    of:
    \begin{itemize}
        \item a carrier set $X : \hSet$,
        \item and an interpretation function $\alpha_X : F_\sigma(X) \to X$.
    \end{itemize}
\end{definition}

In the rest of the dissertation, we use a normal alphabet
(e.g. $X$) to denote the carrier set, and the alphabet in Fraktur font
(e.g. $\mathfrak{X}$) to denote an algebra based on the carrier set.

\begin{example}
($\Nat$, +) is a $\term{Mon}$-algebra, with
the carrier set given by $\Nat$ and its interpretation function given by:
\begin{align*}
    \alpha_X(e) & = 0 \\
    \alpha_X(\mult, [a, b]) & = a + b
\end{align*}
\end{example}

\begin{definition}
    A $F_\sigma$-algebra homomorphism $h: X \rightarrow Y$ is a function such that:
% https://q.uiver.app/#q=WzAsNCxbMCwwLCJGX3tcXHNpZ21hfShYKSJdLFsxLDAsIlgiXSxbMCwxLCJGX3tcXHNpZ21hfShZKSJdLFsxLDEsIlkiXSxbMCwxLCJcXGFscGhhX1giXSxbMCwyLCJGX3tcXHNpZ21hfShoKSIsMl0sWzIsMywiXFxhbHBoYV9ZIiwyXSxbMSwzLCJoIl1d
\[\begin{tikzcd}[ampersand replacement=\&,cramped]
	{F_{\sigma}(X)} \& X \\
	{F_{\sigma}(Y)} \& Y
	\arrow["{\alpha_X}", from=1-1, to=1-2]
	\arrow["{F_{\sigma}(h)}"', from=1-1, to=2-1]
	\arrow["{\alpha_Y}"', from=2-1, to=2-2]
	\arrow["h", from=1-2, to=2-2]
\end{tikzcd}\]
\end{definition}

Informally, this means that for any $n$-ary function symbol $(f, [x_1, \dots, x_n]) : F_{\sigma}(X)$,
\begin{align*}
    h(\alpha_X(f, [a_1, \dots, a_n])) = \alpha_Y(f, [h(x_1), \dots, h(x_n)])
\end{align*}

$F_{\sigma}$-algebras and their morphisms form a category $F_\sigma$-Alg.
For brevity, we also refer to this category as $\sigma$-Alg. The category
is given by the identity homomorphism and composition of homomorphisms.

\section{Free Algebra}\label{sec:universal-algebra:free-algebras}
The category $\sigma$-Alg admits a forgetful functor
$\sigma-\term{Alg} \to \Set$, which forgets all the algebraic structure
and only retains the underlying carrier set.
We use $U$ to denote the forgetful functor, and under our notation
$U(\mathfrak{X})$ is simply $X$.

The free $\sigma$-algebra $\mathfrak{F}$ is simply defined as the left adjoint to $U$.
More concretely, it is given by:
\begin{itemize}
    \item a type constructor $F : \Set \to \Set$,
    \item a universal generators map $\eta_X : X \to F(X)$, such that
    \item for any $\sigma$-algebra $\mathfrak{Y}$, post-composition with $\eta_X$ is an equivalence:
\end{itemize}
\[
\begin{tikzcd}[ampersand replacement=\&]
	\mathfrak{F}(X) \\
	\\
	\mathfrak{Y}
	\arrow["f", dotted, from=1-1, to=3-1]
\end{tikzcd}
\mapsto
\begin{tikzcd}[ampersand replacement=\&]
	X \&\& {F(X)} \\
	\\
	\&\& {Y}
	\arrow["\eta_X", color={rgb,255:red,255;green,54;blue,51}, from=1-1, to=1-3]
	\arrow["f \comp \eta_X"', from=1-1, to=3-3]
	\arrow["{f}", dotted, from=1-3, to=3-3]
\end{tikzcd}\]

More concretely, let $f$ be a morphism from $\mathfrak{F}(X) \to \mathfrak{Y}$.
$f$ would be a homomorphism since $\mathfrak{F}(X)$ and $\mathfrak{Y}$ are
objects in the category $\sigma$-Alg. 
We use
$\blank \comp \eta_X : (\mathfrak{F}(X) \to \mathfrak{Y}) \to (X \to Y)$
to denote the act of turning a homomorphism to a set function by
post-composition with $\eta_X$.
The universal property of free algebras state that $\blank \comp \eta_X$
is a equivalence, i.e. there is a one-to-one mapping from
homomorphisms $\mathfrak{F}(X) \to \mathfrak{Y}$ to set functions $X \to Y$.
Naturally, there would be an inverse mapping which maps set functions $X \to Y$
to homomorphisms $\mathfrak{F}(X) \to \mathfrak{Y}$. We refer to this
as the extension operation $\ext{(\blank)} : (X \to Y) \to (\mathfrak{F}(X) \to \mathfrak{Y})$,
and we use $\ext{f}$ to denote a homomorphism given by a set function $f : X \to Y$.

By definition, if $F : \Set \to \Set$ satisfies the universal property of
free $\sigma$-algebra, then $\mathfrak{F}$ would be the free $\sigma$-algebra.
We use "the" to refer to $F$ since $F$ is unique up to isomorphism. 

\begin{lemma}\label{lem:free-algebras-unique}
Any two free $\sigma$-algebra on the same set are isomorphic.
\end{lemma}
\begin{proof}
Let $E(X)$ and $F(X)$ be the carrier sets of two free $\sigma$-algebras.
This gives us $\eta_1 : X \to E(X)$ and $\eta_2 : X \to F(X)$.
We can extend them to $\ext{\eta_1} : \mathfrak{F}(X) \to \mathfrak{E}(X)$
and  $\ext{\eta_2} : \mathfrak{E}(X) \to \mathfrak{F}(X)$. By uniqueness,
$\ext{\eta_1}$ and $\ext{\eta_2}$ must be the identity homomorphism,
therefore $E(X)$ and $F(X)$ are isomorphic.
\end{proof}

\subsection{Construction}
We can construct the carrier set of the free $\sigma$-algebra
$\mathfrak{F}(X)$ as an inductive type of trees $Tr_\sigma(V)$:
\begin{code}
data Tree (V : UU) : UU where
    leaf : V -> Tree V
    node : F$_\sigma$(Tree V) -> Tree V
\end{code}

The constructors $leaf$ and $node$ correspond to the generator map
and the interpretation function. Concretely, $Tr_\sigma(V)$ can
be thought of as an abstract syntax tree denoting operations done
an elements of $V$. 

Given a $\sigma$-algebra $\mathfrak{X}$, composing a homomorphism
$f : \mathfrak{F}(V) \to \mathfrak{X}$ with $node$ with turn it
into a function $V \to X$.
We can define an inverse to this operation by lifting a set function
$f : V \to X$ to a homomorphism $\ext{f} : \mathfrak{F}(V) \to \mathfrak{X}$
as follow: we first apply $f$ on every leaves of the tree, giving us
$Tr_\sigma(X)$, and evaluate the tree according to the operations
in the nodes, giving us a $X$ as the result.

\begin{example}
A tree for $\sigMon$ with the carrier set $\Nat$ would look like:
\begin{center}
\scalebox{0.7}{
% https://q.uiver.app/#q=WzAsNixbMiwyLCIrIl0sWzIsMywiNCJdLFszLDEsIisiXSxbMiwwLCIxIl0sWzQsMCwiMSJdLFswLDAsIjIiXSxbMCwxXSxbMiwwXSxbMywyXSxbNCwyXSxbNSwwXV0=
\begin{tikzcd}[ampersand replacement=\&,cramped]
	2 \&\& 1 \&\& 1 \\
	\&\&\& {+} \\
	\&\& {+} \\
	\&\&
	\arrow[no head, from=3-3, to=4-3]
	\arrow[from=2-4, to=3-3]
	\arrow[from=1-3, to=2-4]
	\arrow[from=1-5, to=2-4]
	\arrow[from=1-1, to=3-3]
\end{tikzcd}
}
\hspace{2em}
% https://q.uiver.app/#q=WzAsMixbMCwwLCJlIl0sWzAsMiwiMCJdLFswLDFdXQ==
\begin{tikzcd}[ampersand replacement=\&,cramped]
	e \\
	\\
	\arrow[no head, from=1-1, to=3-1]
\end{tikzcd}
\end{center}
    % Some example trees for $\sigMon$, with leaves drawn from $\{\rwb\}$.
    % %
    % \todo{Expand the signature functor and show how this produces binary trees.}
\end{example}

\section{Equation}\label{sec:universal-algebra:equations}
We have shown how to define the operations of an algebra, but we have not
shown how to define the equations of an algebra. For example, a monoid
should satisfy the unit laws and associative law. We now explain how
we can define equations.

Much like operation signatures, every algebra also has an equation signature
$\epsilon$, which is a (dependent) tuple of:
\begin{itemize}
    \item a set of equation symbols $eq : \Set$,
    \item and an arity of free variables $fv : eq \to \Set$.
\end{itemize}

Here, $eq$ gives us a set of names to refer to an equation which
the algebra must satisfy, and the arity function $fv$ specifies the 
number of free variables in the equation, using the cardinality of the
output set as the arity. Once again, other definitions might opt to
define this as $fv : eq \to \Nat$, but we opted to use $\hSet$ and
cardinality instead to allow for infinitary equations.

\begin{example}
We can define the equation signature of monoid as follow: the algebra
$\Mon$ has equation symbols $\{\term{unitl}, \term{unitr}, \term{assocr}\}$,
referring to the unit left law, the unit right law, and the associativity law.
The arity function is defined as:
$fv : \{ \term{unitl} \mapsto \Fin[1], \term{unitr} \mapsto \Fin[1], \term{assocr} \mapsto \Fin[3] \}$.
\end{example}

We can the define a system of equations (or equational theory $T_\epsilon$)
by a pair of trees on the set of free variables, representing the
left hand side and right hand side of the equations.
\begin{definition}
A system of equations $T_\epsilon$ is defined as
$l, r : (e : eq) \to Tr_\sigma(fv(e))$.
\end{definition}

We say a $\sigma$-structure $\mathfrak{X}$ satisfies $T$
iff for all $e : eq$ and
$\rho : fv(e) \to X$, $\ext{\rho}(l(e)) = \ext{\rho}(r(e))$.
Concretely, this means that $\rho : fv(e) \to X$ assigns
free variables in equations to elements of $X$, and
$\ext{\rho} : \mathfrak{F}(fv(e)) \to \mathfrak{X}$ evaluates
a tree given an assignment of free variables. Because $\ext{\rho}$
is a homomorphism, it must evaluate correctly according to the
algebraic structure of $F_\sigma$.

\begin{example}
To show ($\Nat$, 0, +) satisfies monoidal laws, we need to prove
the following properties:
\begin{align*}
\term{unitl}  & : \forall (\rho : \Nat^{\Fin[1]}). \, \rho(0) + 0 \id \rho(0) \\
\term{unitr}  & : \forall (\rho : \Nat^{\Fin[1]}). \, 0 + \rho(0) \id \rho(0) \\
\term{assocr} & : \forall (\rho : \Nat^{\Fin[3]}). \, (\rho(0) + \rho(1)) + \rho(2) \id \rho(0) + (\rho(1) + \rho(2))
\end{align*} 
\end{example}

\section{Construction}\label{algebra:tree}
To show that a construction $F(X)$ is a free $\sigma$-algebra, we first have to
show it is a $\sigma$-algebra by defining how the operations of $\sigma$
would compute on $F(X)$. For example, identity operation for $\List(X)$
is given by a constant function $\lambda().\,[].$, and the $\mult$ operation
would be given by the concatenation function. We then have to show that
it is a free $\sigma$-algebra by defining what the universal map
$\eta : X \to F(X)$ would be, and define an extension operation
$\ext{(\blank)} : (X \to Y) \to (\mathfrak{F}(X) \to \mathfrak{F}(Y))$
would be. Finally, we have to show that $\ext{(\blank)}$ is the inverse
to $\blank \comp \eta$, therefore $F(X)$ does satisfy the universal property
of free algebra.

We note that to actually show $Tr_\sigma(X)$ is a free $\sigma$-algebra,
we need to quotient $Tr_\sigma(X)$ by the equations. To do this for
infinitary operations would require axiom of choice~\cite{blassWordsFreeAlgebras1983},
therefore we can not give a general construction of free algebra as a quotient
without assuming non-constructive principles. Alternatively, we can give
a general construction as a HIT~\cite{univalentfoundationsprogramHomotopyTypeTheory2013},
however doing this in Cubical Agda led to termination checking issues.
After a week on trying to give a general construction, we decided to
move on to other issues instead, therefore we do not give a general
construction in the formalization.

In the next sections, we give constructions of 
free monoids and free commutative monoids, and prove these constructions
are correct under this framework.